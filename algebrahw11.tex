\documentclass[12pt]{article}
\setlength\headheight{14.5pt}
\title{Homework}
\author{Frederick Robinson}
\date{8 March 2010}
\usepackage{amsfonts}
\usepackage{fancyhdr}
\usepackage{amsthm}
\usepackage{setspace}
\doublespacing
\pagestyle{fancyplain}

\begin{document}

\lhead{Frederick Robinson}
\rhead{Math 331: Algebra}

   \maketitle

\setcounter{tocdepth}{2} 

\tableofcontents

\section{Chapter 12 Section 2}
\subsection{Problem 4}

\subsubsection{Question}
Prove that two $3 \times 3$ matrices are similar if and only if they have the same characteristic and same minimal polynomial. Give an explicit counterexample to this assertion for $4 \times 4$ matrices.
\subsubsection{Answer}
We know that two matrices are similar if and only if their list of invariant factors is the same. The characteristic polynomial, say $c$, is the product of all the invariant factors while the minimal polynomial ($m$) is the largest invariant factor. 

For a $3 \times 3$ matrix the characteristic polynomial has degree 3. 

We will divide into cases based on the the degree of the minimal polynomial, showing in each case that the list of invariant factors is uniquely determined by the minimal polynomial and the characteristic polynomial.

\emph{Case 1(Degree 1) : } The minimal polynomial is of the form $m = x - b$ for some $b$. Moreover, since this is the largest invariant factor, the characteristic polynomial must be of the form $c = (x-b)(x-a_1)(x-a_2)$. So the invariant factors are uniquely determined.

\emph{Case 2(Degree 2) : } The minimal polynomial is a quadratic. Moreover, the characteristic polynomial is just $(x^2-ax+b)(x-d)$ and the invariant factors are $c$ and $(x-d)$. So, the invariant factors are uniquely determined by $c$ and $m$.

\emph{Case 3(Degree 3) : } The minimal polynomial in this case is the only invariant factor. Again, the invariant factors are uniquely determined by $c$ and $m$.

Conversely, if the minimal polynomial, and characteristic polynomial of two matrices are not the same then clearly they cannot have the same list of invariant factors. In particular, if they had the same list of invariant factors then the product of this list (the characteristic polynomial) would be the same for each.

To construct a counterexample we seek two sets of four invariant factors whose product is the same and whose largest element agrees but which are themselves different. Consider for instance
\[ (x^2,x,x) \quad \mathrm{and} \quad (x^2,x^2)
\]
Clearly these fulfill the above criteria. So, by construction the invariant factors of the below matrices are different, even though their minimal polynomials and characteristic polynomials differ.

\[\left[\begin{array}{cccc} 
0&0&0&0\\
0&0&0&0\\
0&0&0&0\\
0&0&1&0\\
\end{array} \right] \quad \mathrm{and} \quad 
\left[\begin{array}{cccc} 
0&0&0&0\\
1&0&0&0\\
0&0&0&0\\
0&0&1&0\\
\end{array} \right] 
\]

\subsection{Problem 11}

\subsubsection{Question}
Find all similarity classes of $6 \times 6$ matrices over $\mathbb{C}$ with characteristic polynomial $(x^4-1)(x^2-1)$.
\subsubsection{Answer}
This is essentially the same as listing the invariant factor decompositions. 

\begin{tabular}{ll}
$(x^2-1) (x^4-1)$ &
$\left[ \begin{array}{cccccc}
0&&&&&-1\\
 1&&&&&0\\
 &1&&&&1\\
 &&1&&&0\\
 &&&1&&1\\
 &&&&1&0\\
  \end{array} \right]$ \\ \\
$(x-1), (x-1)(x+1)^2(x^2+1)$ &
$\left[ \begin{array}{cccccc}
1\\
 &0&0&0&0&1\\
 &1&0&0&0&1\\
 &0&1&0&0&0\\
 &0&0&1&0&0\\
 &0&0&0&1&-1\\
  \end{array} \right]$ \\ \\
$(x+1), (x+1)(x-1)^2(x^2+1)$ &
$\left[ \begin{array}{cccccc}
-1\\
 &0&0&0&0&-1\\
 &1&0&0&0&1\\
 &0&1&0&0&0\\
 &0&0&1&0&0\\
 &0&0&0&1&1\\
  \end{array} \right]$ \\ \\
  $(x-1)(x+1), (x-1)(x+1)(x^2+1)$ &
$\left[ \begin{array}{cccccc}
0&1\\
1&0\\
 &&0&0&0&1\\
 &&1&0&0&0\\
 &&0&1&0&0\\
 &&0&0&1&0\\
  \end{array} \right]$ \\ \\
\end{tabular}

\subsection{Problem 18}

\subsubsection{Question}
Let $V$ be a finite dimensional vector space over $\mathbb{Q}$ and suppose $T$ is a nonsingular linear transformation of $V$ such that $T^-1 = T^2 +T$. Prove that the dimension of $V$ is divisible by 3. If the dimension of $V$ is precisely 3 prove that all such transformations $T$ are similar.
\subsubsection{Answer}
First observe that we have 
\[ T^{-1} =T^2+T \Rightarrow T T^{-1} = I  = T^3 + T^2 \]
Moreover, since $x^3+x^2 - 1 $ is irreducible over $\mathbb{Q}$ the only invariant factor of $T$ is just $x^3+x^2 - 1 $ since any invariant factor of $T$ has to divide $x^3+x^2 - 1 $. Hence, the characteristic polynomial must be some power of this invariant factor 
\[ (x^3+x^2 -1)^l \]
and therefore the dimension is $\dim{V}=3l$.

If the dimension is precisely $3$ then the characteristic polynomial must be exactly $ (x^3+x^2 -1)^1 =  x^3+x^2 -1 $ and by definition of similarity any two such $T$ must be similar.


\section{Chapter 12 Section 3}
\subsection{Problem 1}

\subsubsection{Question}
Suppose the vector space $V$ is the direct sum of cyclic $F[x]$-modules whose annihilators are $(x+1)^2$, $(x-1)(x^2+1)^2$, $(x^4-1)$ and $(x+1)(x^2-1)$. Determine the invariant factors and elementary divisors for $V$.
\subsubsection{Answer}
There are 3 cases 
\begin{enumerate}


\item \emph{$F  \neq \mathbb{Z}/2\mathbb{Z}$ (or any product thereof) and $x^2+1$ irreducible:}

In this case we have that 
\[V \cong F[x]/(x^2+1)\oplus F[x]/(x-1)(x^2+1)^2\]
\[ \oplus F[x]/(x+1)(x-1)(x^2+1)\oplus F[x] / (x-1)(x+1)^2 \]
\[\cong (F[x] /(x-1)^3 \oplus F[x]/(x+1)\]
\[ \oplus (F[x]/(x+1)^2)^2 \oplus F[x]/(x^2+1) \oplus F[x]/(x^2+1)^2\]
so we observe that the elementary divisors are just 
\[(x-1),(x+1), (x+1)^2, x^2+1, (x-1),  (x^2+1)^2, (x-1), (x+1)\]
and the invariant factors are
\[ (x-1)(x+1)^2(x^2+1), (x-1)(x+1)^2(x^2+1)^2, (x-1)(x+1)\]

\item \emph{$F  \neq \mathbb{Z}/2\mathbb{Z}$ (or any product thereof)  with $x^2+1$  reducible:}
In this case we have the same as above, but with a few changes due to the fact we can now factor $x^2+1$

The elementary divisors are just 
\[(x-1), (x-1),(x+1), (x+1)^2, (x-1), (x+1)^2,  (x+1), (x-i), (x+i)^2, (x-i)^2 \]
and the invariant factors are
\[ (x-1)(x+1)^2(x+i)^2(x-i)^2, (x-1)(x+1)^2(x+i)(x-i), (x-1)(x+1)\]



\item \emph{$F  = \mathbb{Z}/2\mathbb{Z}$ (or some product thereof):} 

\[V=F[x]/(x+1)^2 \oplus F[x]/(x+1)^3\oplus F[x]/(x+1)^4\oplus F[x]/(x+1)^5\]
which implies that the elementary factors and invariant factors are the same. They are in particular just 
\[(x+1)^2, (x+1)^3, (x+1)^4, (x+1)^5\]
\end{enumerate}

\subsection{Problem 18}

\subsubsection{Question}
Determine all possible Jordan canonical forms for a linear transformation with characteristic polynomial $(x-2)^3(x-3)^2$.
\subsubsection{Answer}
There are several possibilities for invariant factors of a matrix which has the above specified characteristic polynomial. In particular the following:
\[ (x-2)^3(x-3)^2\]\[(x-2)^2(x-3)^2,(x-2) \]\[(x-2)^3(x-3),(x-3)\]\[(x-2)(x-3)^2,(x-2),(x-2) \]\[ (x-2)^2(x-3),(x-3)(x-2)\]\[(x-2)(x-3),(x-2)(x-3),(x-2) \]

\[\left[ \begin{array}{ccc|cc} 
2&1&&& \\
&2&1&& \\
&&2&& \\
\hline
&&&3&1 \\
&&&&3 \\
\end{array} \right] \quad \left[ \begin{array}{cc|cc|c} 
2&1&&& \\
&2&&& \\
\hline
&&3&1& \\
&&&3& \\
\hline
&&&&2 \\
\end{array} \right] \quad \left[ \begin{array}{ccc|c|c} 
2&1&&& \\
&2&1&& \\
&&2&& \\
\hline
&&&3& \\
\hline
&&&&3 \\
\end{array} \right] \]\[\left[ \begin{array}{c|cc|c|c} 
2&&&& \\
\hline
&3&1&& \\
&&3&& \\
\hline
&&&2& \\
\hline
&&&&2 \\
\end{array} \right] \quad \left[ \begin{array}{cc|c|c|c} 
2&1&&& \\
&2&&& \\
\hline
&&3&& \\
\hline
&&&3& \\
\hline
&&&&2 \\
\end{array} \right] \quad \left[ \begin{array}{c|c|c|c|c} 
2&&&& \\
\hline
&3&&& \\
\hline
&&2&& \\
\hline
&&&3& \\
\hline
&&&&2 \\
\end{array} \right] \]


\end{document}
