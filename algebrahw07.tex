\documentclass[12pt]{article}
\setlength\headheight{14.5pt}
\title{Homework}
\author{Frederick Robinson}
\date{6 February 2010}
\usepackage{amsfonts}
\usepackage{fancyhdr}
\usepackage{amsthm}
\usepackage{amssymb}
\usepackage{setspace}
\doublespacing
\pagestyle{fancyplain}

\begin{document}

\lhead{Frederick Robinson}
\rhead{Math 331: Algebra}

   \maketitle

\setcounter{tocdepth}{2} 

\tableofcontents

\section{Chapter 9 Section 1}
\subsection{Problem 4}

\subsubsection{Question}
Prove that the ideals $(x)$  and $(x,y)$ are prime ideals in $\mathbb{Q}[x,y]$ but only the latter ideal is a maximal idea.
\subsubsection{Answer}

I claim that $(x)$ is a prime ideal in $\mathbb{Q}[x,y]$.
\begin{proof}
Let $a b \in (x)$ be the product of two elements in $\mathbb{Q}[x,y]$. We wish to show that either $a \in (x)$ or $b \in (x)$. Since $ab \in (x)$ we know that each term of $ab$ has at least one $x$. Assume towards a contradiction that neither $a\in (x)$ nor $b\in (x)$. Then each of $a$ and $b$ contain at least one term which has no $x$. Hence, their product contains a terms which is the product of these terms and contains no $x$. Contradiction.
\end{proof}

I claim that $(x,y)$ is a prime ideal in $\mathbb{Q}[x,y]$.
\begin{proof}
Let $a b \in (x,y)$ be the product of two elements in $\mathbb{Q}[x,y]$. We wish to show that either $a \in (x,y)$ or $b \in (x,y)$. Since $a b \in (x,y)$ we know that each term of this product must contain an $x$ or a $y$. Assume towards a contradiction that $a \notin (x,y)$ and $b \notin (x,y)$. Then neither has the property that each term contains an $x$ or a $y$. Thus each has at least one term which contains neither $x$s nor $y$s. However, the product of such a term contains neither $x$s nor $y$s. Contradiction.
\end{proof}

I claim that $(x,y)$ is a maximal ideal.
\begin{proof}
Suppose that $(x,y)$ is not a maximal ideal, then there exists some other ideal say $I$ such that $(x,y) \subsetneq I \subsetneq \mathbb{Q}[x,y]$. Since every element of $\mathbb{Q}[x,y]$ which contains only terms with $x$ and $y$ is contained in  $(x,y)$ it must be that $I$ contains at least one element say $k$ such that $k$ has at least one term with neither $x$s nor $y$s. However $I \subset (x,y,k)=\mathbb{Q}[x,y] $. In particular since $(x,y,k)$ is an ideal it must be that it is closed under addition. Furthermore there exists an element of $(x,y)$ say $l$ whose terms containing $x$ and $y$ are the same as those of $k$ but with opposite sign since $(x,y)$ contains each element of $\mathbb{Q}[x,y]$ whose terms each have $x$ and $y$. This means that $k+l \in (x,y,k) = c$ for some constant term $c$. Then, since an ideal must be closed under multiplication by elements of the ring and since $\mathbb{Q}$ is closed under multiplicative inverse $1 \in (x,y,k) \Rightarrow (x,y,k)=\mathbb{Q}[x,y]$  
\end{proof}

I claim that $(x)$ is not a maximal ideal.
\begin{proof}
First observe that $(x) \subset (x,y)$ by definition of finitely generated ideals. Moreover I claim that $(x) \neq (x,y)$. In particular though $y$ is an element of the latter ideal it is not in the former.
\end{proof}

\subsection{Problem 5}

\subsubsection{Question}
Prove that $(x,y)$  and $(2,x,y)$ are prime ideals in $\mathbb{Z}[x,y]$ but only the latter ideal is a maximal ideal.
\subsubsection{Answer}

I claim that $(x,y)$ is a prime ideal in $\mathbb{Z}[x,y]$.
\begin{proof}
Let $a b \in (x,y)$ be the product of two elements in $\mathbb{Z}[x,y]$. We wish to show that either $a \in (x,y)$ or $b \in (x,y)$. Since $a b \in (x,y)$ we know that each term of this product must contain an $x$ or a $y$. Assume towards a contradiction that $a \notin (x,y)$ and $b \notin (x,y)$. Then neither has the property that each term contains an $x$ or a $y$. Thus each has at least one term which contains neither $x$s nor $y$s. However, the product of such a term contains neither $x$s nor $y$s. Contradiction.
\end{proof}

I claim that  $(2,x,y)$ is a prime ideal in $\mathbb{Z}[x,y]$.
\begin{proof}
Let $a b \in (2,x,y)$ be the product of two elements in $\mathbb{Z}[x,y]$. We wish to show that either $a \in (2,x,y)$ or $b \in (2,x,y)$. Since $a b \in (x,y)$ we know that each term of this product must contain an $x$ or a $y$ except for possibly a constant term which must be a multiple of 2 if it is present. Assume towards a contradiction that $a \notin (x,y)$ and $b \notin (x,y)$. Then neither has the property that each term contains an $x$ or a $y$ except for a constant term which must be a multiple of 2 if present. Thus each has a constant term which is not a multiple of 2. However, the product of these terms is not 2 since in $\mathbb{Z}$ the only numbers whose product is a multiple of 2 are themselves multiples of 2 (or 1, but if this is the case then the other must be a multiple of 2). Contradiction.
\end{proof}

I claim that $(2,x,y)$ is a maximal ideal in  $\mathbb{Z}[x,y]$.
\begin{proof}
Suppose towards a contradiction that there exists some ideal say $I$ such that $(2,x,y ) \subsetneq I \subsetneq \mathbb{Z}[x,y]$. In particular $I$ must contain some element of $\mathbb{Z}[x,y]$ which is not in $(2,x,y)$. Since each element of $\mathbb{Z}$ which has no constant term is in $(2,x,y)$, our new element of $I$ must have a constant term. Also, since each member of $\mathbb{Z}[x,y]$ whose constant term is a multiple of 2 is a member of $(2,x,y)$ the constant term of our new element must not be a multiple of 2. So, we have established that there exists $k \in I $ such that $k$ has a constant term which is of the form $k=2n+1$ for some $n \in \mathbb{Z}$. Now, since $I$ is a superset of $(2,x,y)$ and must be closed under addition (by definition of an ideal) it must also contain just $k$ since in $(2,x,y)$ there is an element whose nonconstant terms are the same as those of $k$ but with opposite sign. Similarly by closure under addition $1 \in I$ since $-2n \in (2,x,y) \Rightarrow 2n+1-2n = 1 \in I$. Thus, $I = \mathbb{Z}[x,y]$.
\end{proof}

I claim that $(x,y)$ is not a maximal ideal.
\begin{proof}
By definition of finitely generated ideals we have that $(x,y) \subset (2,x,y)$. Furthermore since in particular $2 \in (2,x,y)$ but $ 2 \notin (x,y)$ we have $(x,y) \neq (2,x,y)$. Thus, $(x,y)\subset (2,x,y) \subset \mathbb{Z}[x,y]$ and $(x,y)$ is not a maximal ideal.
\end{proof}

\subsection{Problem 6}

\subsubsection{Question}
Prove that $(x,y)$ is not a principal ideal in $\mathbb{Q}[x,y]$.
\subsubsection{Answer}
I claim that $(x,y)$ is not a principal ideal in $\mathbb{Q}[x,y]$.
\begin{proof}
In particular there exist no element $k$ such that $(k)=(x,y)$. For suppose towards a contradiction that there exists such a $k$. Then, each member of $(x,y)$ may be written as a product of $k$ and a member of $\mathbb{Q}[x,y]$. In particular since $x,y\in (x,y)$ we have that $x=k \cdot l $ and $y=k \cdot m$. However this implies that $k \in \mathbb{Q}$, but $(x,y)$ contains no constant polynomials. Contradiction.
\end{proof}

\section{Chapter 9 Section 2}
\subsection{Problem 1}

\subsubsection{Question}
Let $f(x) \in F[x]$ be a polynomial of degree $n \geq 1$ and let bars denote passage to the quotient $F[x]/(f(x))$. Prove that for each $\overline{g(x)}$ there is a unique polynomial $g_0(x)$ of degree $\leq n-1$ such that $\overline{g(x)} = \overline{g_0(x)}$ (equivalently, the elements $\overline{1}, \overline{x}, \dots, \overline{x^{n-1}}$ are a \emph{basis}  of the vector space $F[x]/(f(x))$ over $F$ -- in particular, the dimension of this space is $n$). [Use the Division Algorithm.]
\subsubsection{Answer}

I claim that for each $\overline{g(x)}$ there is a unique polynomial $g_0(x)$ of degree $\leq n-1$ such that $\overline{g(x)} = \overline{g_0(x)}$
\begin{proof}
Let $g(x)$ be a polynomial over $F$ of degree $m>n$ and maximal coefficient is $z$. Then, if we take the sum $g(x) - z x^{m-n}  f(x)$ we will get an element of the same equivalence class mod $f(x)$ as $g(x)$ since we are just subtracting by a multiple of $f(x)$. Since we can repeat this as long as the degree of $g(x)$ is greater than or equal to that of $f(x)$ we can use this method to generate a representative $g_0(x)$ of $\overline{g(x)}$ which has degree  $\leq n-1$. 

Moreover $g_0(x)$ is the unique such polynomial with degree $\leq n-1$ for any two polynomials $h(x)$ and $j(x)$ whose degree are is less than that of $f(x)$ are distinct mod $f(x)$. Suppose not, then there would exist an element $l(x) \in (f(x))$ such that $l(x)+j(x)=h(x)$ However each potential $l(x)$ can be written as $f(x) t(x)=l(x)$ for some nonzero $t(x) \in F[x]$. So we get $f(x)t(x)+j(x)=h(x)$. However, any product of $f(x)$ with some other nonzero polynomial has degree at least that of $f(x)$ but then $N(f(x)t(x))\geq n$ whereas $N(h(x))\leq n-1$. Contradiction.
\end{proof}

\subsection{Problem 2}

\subsubsection{Question}
Let $F$ be a finite field of order $q$ and let $f(x)$ be a polynomial in $F[x]$ of degree $n \geq 1$. Prove that $F[x]/(f(x))$ has $q^n$ elements. [Use the preceding exercise.]
\subsubsection{Answer}

I claim that if $F$ is a finite field of order $q$ and $f(x)$ a polynomial in $F[x]$ of degree $n \geq 1$ then $F[x]/(f(x))$ has $q^n$ elements.
\begin{proof}
Let $g(x) \in F[x]$. By the previous exercise $\overline{g(x)}$ has a unique representative in $F[x]/(f(x))$ with degree $\leq n-1$. There are $q^n$ polynomials of degree strictly less than $n$. Each $x^n$ term has one of the $q$ members of $F$ as its coefficient. Since there are $q^n$ such elements there are $q^n$ distinct equivalence classes in $F[x]/(f(x))$. Each equivalence class corresponding to a given polynomial of degree $\leq n-1$ is nonempty since in particular it contains the representative polynomial.
\end{proof}

\subsection{Problem 7}

\subsubsection{Question}
Determine all the ideals of the ring $\mathbb{Z}[x]/(2,x^3+1)$.
\subsubsection{Answer}

By an isomorphism theorem we can write
\[\mathbb{Z}[x]/(2,x^3+1) \cong (\mathbb{Z}[x]/2)/((2,x^3+1)/(2)) \cong \mathbb{Z}/2\mathbb{Z}[x]/(x^3+1)\]
\[\cong \mathbb{Z}/ 2 \mathbb{Z} / ((x+1)(x^2+x+1))\]
and by Proposition 16 this is just
\[\cong \mathbb{Z}/ 2 \mathbb{Z} / (x+1) \times  \mathbb{Z}/ 2 \mathbb{Z} / (x^2+x+1)\]
So the ideals of this ring are just multiples of $\cong \mathbb{Z}/ 2 \mathbb{Z} / (x+1) $ and $  \mathbb{Z}/ 2 \mathbb{Z} / (x^2+x+1)$
\subsection{Problem 8}

\subsubsection{Question}
Determine the greatest common divisor of $a(x) = x^3 -2$ and $b(x)=x+1$ in $\mathbb{Q}[x]$ and write it as a linear combination (in $\mathbb{Q}[x]$) of $a(x)$ and $b(x)$.
\subsubsection{Answer}
Both $a(x) $ and $b(x)$ are irreducible. The former is irreducible since it has no roots in  $\mathbb{Q}$, and the latter because it has no constant rational divisors. Thus, by proposition 13 page 287 their greatest common factor is just 1. We can write 1 as a linear combination in $\mathbb{Q}[x]$ as 
\[1= \frac{1}{3} (x+1) (x^2-x+1)-1(x^3-2)\]

\section{Chapter 9 Section 4}
\subsection{Problem 1}

\subsubsection{Question}
Determine whether the following polynomials are irreducible in the rings indicated. For those that are reducible, determine their factorization into irreducibles. The notation $\mathbb{F}_p$ denotes the finite field $\mathbb{Z}/p\mathbb{Z}$, $p$ a prime.
\begin{enumerate}
\item $x^2+x+1$ in $\mathbb{F}_2[x]$.
\item $x^3+x+1$ in $\mathbb{F}_3[x]$.
\item $x^4+1$ in $\mathbb{F}_5[x]$.
\item $x^4+10x^2+1$ in $\mathbb{Z}[x]$.
\end{enumerate}
\subsubsection{Answer}
\begin{enumerate}
\item $x^2+x+1$ is irreducible in $\mathbb{F}_2[x]$ since it has no roots and its coeffients have no multiple in common.
\item $x^3+x+1 =(x-1)(x^2+x+2)$ in $\mathbb{F}_3[x]$.
\item $x^4+1$ has no roots in $\mathbb{F}_5[x]$. Thus any factorization of $x^4+1$ must be as a product of order 2 terms. Moreover  since the only pairs of  elements of $\mathbb{F}_5$ whose product is 1 are 2,3 and 4,4 and 1,1 there are but a few possibilities for factorizations. Checking each as in the next part reveals that there are no such factorizations. 
\item $x^4+10x^2+1$ has no roots in $\mathbb{Z}[x]$ since it is monic and by Proposition 11, $1+10+1 =12 \neq 0 $ implies it has no roots. Thus, if it is reducible it must be as a product of order 2 terms. However, this being the case each such term must be of the form $\pm x^2+a x \pm 1$ for some $a$ since in $\mathbb{Z}$ the only numbers whose product is 1 are 1 and -1. Thus we must have a product of one of the following forms:
\[1+a x+b x-2 x^2+a b x^2-a x^3-b x^3+x^4\]
\[1-a x-b x-2 x^2+a b x^2+a x^3+b x^3+x^4\]
\[1-a x-b x+2 x^2+a b x^2-a x^3-b x^3+x^4\]
\[1+a x+b x+2 x^2+a b x^2+a x^3+b x^3+x^4\]
of course up to change in $a$ and $b$ the only of these which are unique are 
\[1-a x-b x-2 x^2+a b x^2+a x^3+b x^3+x^4\]
and
\[1+a x+b x+2 x^2+a b x^2+a x^3+b x^3+x^4\]
If the former were the case then we would have $ab-2=10$ and $-a-b=0$ and $a+b=0$. This is a contradiction however since the $-a-b=0 \Rightarrow a=b=0$ but $0-2\neq 10$. If the latter were the case we would get a contradiction as well since we would have $a+b=0$ and $2+ab=10$ and $a+b=0$. The first equation implies that $ a b \leq 0$ but this is inconsistent with the second equation. Thus $x^4+10x^2+1$ is irreducible in $\mathbb{Z}[x]$
\end{enumerate}


\section{Chapter 9 Section 5}
\subsection{Problem 5}

\subsubsection{Question}
Let $\varphi$ denote Euler's $\varphi$-function. Prove the identity $\sum_{d|n}\varphi(d) = n$, where the sum is extended over all the divisors $d$ of $n$. [First observe that the identity is valid when $n=p^m$ is the power of a prime $p$ since the sum telescopes. Write $n=p^m n'$ where $p$ does not divide $n'$. Prove that $\sum_{d|n}\varphi (d) = \sum_{d''|p^m}\varphi(d'')\sum_{d'|n'} \varphi(d')$ by multiplying out the right hand side and using the multiplicativity $\varphi(ab)=\varphi(a)\varphi(b)$ when $a$ and $b$ are relatively prime. Use induction to complete the proof. This problem may be done alternatively by letting $Z$ be the cyclic group of order $n$ and showing that because $Z$ contains a unique subgroup of order $d$ for each $d$ dividing $n$, the number of elements of $Z$ of order $d$ is $\varphi(d)$. Then $|Z|$ is the sum of $\varphi(d)$ as $d$ runs over all divisors of $n$.]
\subsubsection{Answer}
Let $Z$ be the cyclic group of order $n$. $Z$ contains a unique subgroup of order $d$ for each $d$ which divides $n$ by the Fundamental Theorem of Cyclic Groups. In particular if $Z = \left< x \right>$ these subgroups are given by the powers of $x^{n/d}$. Since the members of such a  subgroups whose order is $d$ is just $\varphi(d)$ and an element of a group can have order $d$ if and only if it is a member of a cyclic subgroup of order $d$ we see that the number of elements in $Z$ whose order is $d$ is given by $\varphi(d)$. Since the members of a cyclic group are all of some finite order less than that of the group we see in particular that $|Z|=n=\sum_{d|n}\varphi(d)$ as claimed.
\end{document}
