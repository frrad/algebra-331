\documentclass[10pt]{article}
\setlength\headheight{14.5pt}
\title{Homework}
\author{Frederick Robinson}
\date{10 May 2010}
\usepackage{amsfonts}
\usepackage{fancyhdr}
\usepackage{amsthm}
\usepackage{setspace}
%\doublespacing
\pagestyle{fancyplain}

\begin{document}

\lhead{Frederick Robinson}
\rhead{Math 331: Algebra}

   \maketitle

\setcounter{tocdepth}{2} 

%\tableofcontents

\section{Chapter 14 Section 1}
\subsection{Problem 1}
\subsubsection{Question}
\begin{enumerate}
\item Show that if the field $K$ is generated over $F$ by the elements $\alpha_1, \dots, \alpha_n$ then an automorphism $\sigma$ of $K$ fixing $F$ is uniquely determined by $\sigma(\alpha_1), \dots, \sigma(\alpha_n)$. In particular show that an automorphism fixes $K$ if and only if it fixes a set of generators for $K$.
\item Let $G \leq \mathrm{Gal}(K / F)$ be a subgroup of the Galois group of the extension $K/F$ and suppose $\sigma_1, \dots, \sigma_k$ are generators for $G$. Show that the subfield $E/F$ is fixed by $G$ if and only if it is fixed by the generators $\sigma_1, \dots, \sigma_k$
\end{enumerate}
\subsubsection{Answer}
\begin{enumerate}
\item \begin{proof}Let $\alpha_1, \dots, \alpha_n$ be a set of generators for $K/F$. Any element of $K$ can therefore be expressed uniquely in the form $c_0 + c_1\alpha_1 + \dots + c_n \alpha_n$ with $c_i \in F$. An automorphism of $K$ which fixes $F$ must take $c_i \mapsto c_i$ for any $c_i \in F$ since it fixes $F$. Therefore, by properties of homomorphism it must take $c_0 + c_1\alpha_1 + \dots + c_n \alpha_n \mapsto c_0 + c_1 \sigma(\alpha_1) + \dots + c_n \sigma(\alpha_n)$ and the automorphism is uniquely determined by $\sigma(\alpha_1), \dots, \sigma(\alpha_n)$ as claimed.\end{proof}

If an automorphism fixes $K$ then $\alpha_i \mapsto \alpha_i$ via $\sigma$ for any basis element $\alpha_i$. Conversely, by the foregoing if $\sigma$ fixes a basis for $K/F$ then $\sigma$ fixes all of $K$.

\item \begin{proof} ($\Leftarrow$) If the generators $\sigma_1, \dots, \sigma_k$ fix all of $E/F$ then so does the entire subgroup $G$ since by definition any member of $G$ may be written as a combination of the generators and by properties of homomorphism such combinations fix all of $E/F$ if each does.

Conversely ($\Rightarrow$), if all of $G$ fixes some subfield $E/F$ then it is in particular fixed by generators for $G$ say $\sigma_1, \dots ,  \sigma_k$ as claimed.\end{proof}
\end{enumerate}

\subsection{Problem 4}
\subsubsection{Question}
Prove that $\mathbb{Q}(\sqrt 2)$ and $\mathbb{Q} (\sqrt{3})$ are not isomorphic.
\subsubsection{Answer}
Any isomorphism $\varphi$ is completely determined by where it takes a set of generators for $\mathbb{Q}(\sqrt2)$ in particular it must be of the form $a + b \sqrt 2 \mapsto a(c + d \sqrt3)+ b (e + f \sqrt3)$ where $c+d \sqrt 3$ is the image of 1 under $\varphi$ and $\varphi(\sqrt2)=e+f \sqrt3$. Furthermore, we must have $\varphi(1) =1 \Rightarrow c=1, d=0$, so any isomorphism must be of the form $a+ b \sqrt2 \mapsto (a+be)+bf\sqrt3)$ for some $e,f \in \mathbb{Q}$. 

However, in order that $\varphi$ be a homomorphism we must have $\varphi(x^2)= \varphi(x)^2$. Some computation reveals that for an arbitrary element $x  = a +b\sqrt2$ we get $\varphi(x^2) = \varphi(a^2 + 2b^2+ 2ab\sqrt2)=a^2 + 2b^2+ 2ab e +2ab f \sqrt3  $ whereas $\varphi(x)^2 = ((a+be )+ bf\sqrt3)^2= (a+be)^2 +3bf + (2bfa+2b^2ef)\sqrt3 $. In order for this equality to hold we must have in particular $2bfa+2b^2ef = 2abf$ with $a,b$ arbitrary. Thus, $fa+bef = 2af \Rightarrow bef = af \Rightarrow be=a $, a contradiction since $a, b$ were arbitrary.

\subsection{Problem 5}
\subsubsection{Question}
Determine the automorphisms of the extension $\mathbb{Q}(\sqrt[4]2)/ \mathbb{Q}(\sqrt2)$ explicitly.
\subsubsection{Answer}
The automorphisms of  $\mathbb{Q}(\sqrt[4]2)/ \mathbb{Q}(\sqrt2)$  are completely determined by where they take $\sqrt[4]2$. However, since the subfield $\mathbb{Q}(\sqrt2)$ is fixed (in particular the element $\sqrt2$ ) we have $(\sqrt[4]2)^2= \varphi((\sqrt[4]2)^2)=(\varphi(\sqrt[4]2))^2$. Thus $\varphi(\sqrt[4]2) = \pm \sqrt[4]2$ are the only automorphisms of $\mathbb{Q}(\sqrt[4]2)/ \mathbb{Q}(\sqrt2)$.


\subsection{Problem 10}
\subsubsection{Question}
Let $K$ be an extension of the field $F$. Let $\varphi : K \to K'$ be an isomorphism of $K$ with a field $K'$ which maps $F$ to the subfield $F'$ of $K'$. Prove that the map $\sigma \mapsto \varphi \sigma \varphi^{-1}$ defines a group isomorphism $\mathrm{Aut}(K/ F) \to^{\sim} \mathrm{Aut}(K'/F')$
\subsubsection{Answer}
\begin{proof}
Let $\sigma, \tau$ be arbitrary elements of Aut$(K/F)$. The specified map is a homomorphism since $\varphi (\sigma + \tau ) \varphi^{-1} = \varphi \sigma \varphi^{-1}+ \varphi \tau \varphi^{-1}$ and $\varphi (\sigma \tau) \varphi^{-1} = \varphi \sigma \varphi^{-1} \varphi \tau \varphi^{-1}$. 

It's injective since $\varphi \sigma \varphi^{-1} = \varphi \tau \varphi^{-1} \Rightarrow \varphi \sigma = \varphi \tau \Rightarrow \sigma = \tau$ and surjective since given $\sigma \in $Aut$(K'/F')$ setting $\tau = \varphi^{-1}\sigma\varphi $ we have $\varphi \tau \varphi^{-1} = \varphi \varphi^{-1} \sigma  \varphi \varphi^{-1}= \sigma$.
\end{proof}


\section{Chapter 14 Section 2}
\subsection{Problem 2}
\subsubsection{Question}
Determine the minimal polynomial over $\mathbb{Q}$ for the element $1+\sqrt[3]2+\sqrt[3]4$.
\subsubsection{Answer}
It is easy to check that $\alpha = 1+\sqrt[3]2+\sqrt[3]4$ is a root of the polynomial 
\[ f(x) = x^3 -3 x^2  -3 x - 1\]
Moreover, since $\alpha$ is contained in the degree 3 extension $\mathbb{Q}(\sqrt[3]2)$ its minimal polynomial is of degree 3 or 1. It can't be 1 since $\alpha \notin \mathbb{Q}$ thus the minimal polynomial must have degree 3. Hence, $f$ is the minimal polynomial.

\subsection{Problem 3}
\subsubsection{Question}
Determine the Galois group of $(x^2-2)(x^2-3)(x^2-5)$. Determine \emph{all} the subfields of the splitting field of this polynomial.
\subsubsection{Answer}
The splitting field is $\mathbb{Q}(\sqrt2, \sqrt 3, \sqrt 5)$, an order 8 extension since $2,3,5$ are all prime. The group of automorphisms generated by 
\[\sigma: \sqrt2 \mapsto -\sqrt2\quad \tau: \sqrt3\mapsto -\sqrt3 \quad \varphi:\sqrt5\mapsto -\sqrt5\]
is of order 8, and so this is the entire Galois group. It is isomorphic to $\mathbb{Z}/2\mathbb{Z} \times\mathbb{Z}/2\mathbb{Z} \times\mathbb{Z}/2\mathbb{Z}$.

By Galois correspondence we have subfields of the splitting field in bijective correspondence to subgroups of the Galois group. The degree 4 subfields are in particular
\[\mathbb{Q}(\sqrt2,\sqrt{15}), \mathbb{Q}(\sqrt{10},\sqrt3), \mathbb{Q}(\sqrt6,\sqrt5), \mathbb{Q}(\sqrt3,\sqrt5), \mathbb{Q}(\sqrt2,\sqrt5), \mathbb{Q}(\sqrt2,\sqrt3),\mathbb{Q}(\sqrt{6},\sqrt{10})  \]
The degree 2 subfields are 
\[\mathbb{Q}(\sqrt2), \mathbb{Q}(\sqrt3), \mathbb{Q}(\sqrt5), \mathbb{Q}(\sqrt{15}), \mathbb{Q}(\sqrt{10}), \mathbb{Q}(\sqrt6), \mathbb{Q}(\sqrt{30}).\]


\subsection{Problem 4}
\subsubsection{Question}
Let $p$ be a prime. Determine the elements of the Galois group of $x^p-2$.
\subsubsection{Answer}
The splitting field for $x^p-2$ is generated by $\theta$ the real $\sqrt[p]2$ and $\zeta_p$ a principle $p$th root of unity. Therefore, since $\mathbb{Q}(\theta) \subset \mathbb{R}$ and $x^p-2$ is Eisenstein the splitting field has degree $\varphi(p) \cdot p = p(p-1) = p^2-p$.

A member of the Galois group is completely defined by where it takes these generators. In particular we have the possibilities
\[\left\{ \begin{array}{ll} 
\theta \mapsto \theta\zeta^a & a = 1,2, \dots, p \\
\zeta_p \mapsto (\zeta_p)^a & a = 1, 2, \dots, p-1
\end{array}\right.\]
Since we have already determined that the order of the Galois group is $p^2-p$ and there are exactly $p^2-p$ possibilities all of them are elements of the Galois group 

\end{document}
