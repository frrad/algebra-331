\documentclass[12pt]{article}
\setlength\headheight{14.5pt}
\title{Homework}
\author{Frederick Robinson}
\date{21 February 2010}
\usepackage{amsfonts}
\usepackage{fancyhdr}
\usepackage{amsthm}
\usepackage{setspace}
\doublespacing
\pagestyle{fancyplain}

\begin{document}

\lhead{Frederick Robinson}
\rhead{Math 331: Algebra}

   \maketitle

\setcounter{tocdepth}{2} 

\tableofcontents

\section{Chapter 10 Section 3}
\subsection{Problem 4}
\subsubsection{Question}
An $R$-module $M$ is called a \emph{torsion} module if for each $m \in M$ there is a nonzero element $r \in R$ such that $r m =0$, where $r$ may depend on $m$ (i.e., $M = \mathrm{Tor}(M)$ in the notation of Exercise 8 of Section 1). Prove that every finite abelian group is a torsion $\mathbb{Z}$-module. Give an example of an infinite abelian group that is a torsion $\mathbb{Z}$-module.
\subsubsection{Answer}
\begin{proof}
Let $G$ be a finite abelian group and let $x \in G$ for some $x \neq 0$. I claim that there exists some $x n \in G$ such that $x n= 0$ for $n \neq 0$. Since the entire group is finite there exists some least $x m$ such that $x m=x$ and $m\neq 1$. Thus $ x m = x \Rightarrow xm-x = 0 \Rightarrow x(m-1)=0$. So we have constructed a nonzero $n$ which fulfills the desired criterion. 
\end{proof}

I claim that the infinite direct sum of $\mathbb{Z}/i \mathbb{Z}$ for each $i \in \mathbb{Z}^+$ say $X$ is a torsion $\mathbb{Z}$-module.
\begin{proof}\label{done}
There are only finitely many nonzero components of a given element of $X$. Say that each nonzero element is in position $k_j$ (that is, it is a member of the cyclic group of order $\mathbb{Z}/k_j\mathbb{Z}$). Take the product $k_1k_2\dots k_m$. The product of this number with our element is zero since in any cyclic group the product of a multiple of its order and any element is just zero.
\end{proof} 

\subsection{Problem 6}
\subsubsection{Question}
Prove that if $M$ is a finitely generated $R$-module that is generated by $n$ elements that every quotient of $M$ may be generated by $n$ (or fewer) elements. Deduce that quotients of cyclic modules are cyclic.
\subsubsection{Answer}
\begin{proof}
Assume that $M$ is generated by some set say $\{m_1,m_2,\dots,m_n\}$. Then any element of $M$ can be written in the form $m = r_1m_1+r_2m_2+\cdots + r_nm_n$. 

The quotient module $M/N$ must be finitely generated. In particular the set $\{M_1,M_2,\dots,M_n\}$ generates $M/N$ where  each $M_i \in M/N$ has $m_i \in M_i$ as an arbitary element $x \in M$ can be written as $x = r_1m_1+r_2m_2+\cdots + r_nm_n$ so $ x \in  r_1M_1+r_2M_2+\cdots + r_nM_n$\end{proof}

A module is cyclic if and only if it is generated by one element. Thus, by the above proof if a module is cyclic it is generated by one element and its quotients are also generated by one element and are therefore cyclic too.



\subsection{Problem 7}
\subsubsection{Question}
Let $N$ be a submodule of $M$. Prove that if both $M/N$ and $N$ are finitely generated then so is $M$.
\subsubsection{Answer}
\begin{proof}
We may divide the elements of $M$ into equivalence classes where two elements belong to the same equivalence class if and only if they are brought to the same element of $M/N$ by the natural quotient homomorphism. Moreover we know that any element of one of these equivalence classes say $l \in L$ can be written as $l = n + x$ for some $n \in N$ and $x$ the same value for each $l \in L$. 

Hence, since $N$ is finitely generated, and $M/N$ is finitely generated there is a finite collection of $m_1,m_2, \dots, m_n$ which generate each member of the equivalence classes corresponding to the members of the quotient group which generate it. These are in particular the set of generators for $N$ together with one element of each generator in the quotient module.

So $m_1,m_2, \dots, m_n$ generate $M$ as each element of the quotient module can be written as a product of the generating elements and thence any member of the corresponding equivalence class can be written as a product of members of the equivalence classes corresponding to the generating elements of the quotient module.
\end{proof}

\subsection{Problem 9}
\subsubsection{Question}
An $R$-module $M$ is called \emph{irreducible} if $M \neq 0$ and if $0$ and $M$ are the only submodules of $M$. Show that $M$ is irreducible if and only if $M \neq 0$ and $M$ is a cyclic module with any nonzero element as generator. Determine all the irreducible $\mathbb{Z}$-modules.
\subsubsection{Answer}
Throughout assume $M \neq 0$

We prove ($\Rightarrow$) that if any nonzero element of $M$ generates it then $M$ is irreducible. 
\begin{proof}
Let $N \subset M$ be a submodule of $M$. Suppose that it is nonzero. Then it contains some nonzero element $n \in N$. By assumption $n$ generates $N$. Hence, by closure under product $M= Rn \subset N$. Thus $N=M$ and every nonzero submodule of $M$ is all of $M$.
\end{proof}
We prove ($\Leftarrow$) that  any nonzero element of $M$ generates it if $M$ is irreducible. 
\begin{proof}
Let $x\in M$ be nonzero. Since $x$ is nonzero we know that $\left< x \right> \neq 0$ thus, by irreducibility of $M$  it must be that $\left< x \right> = M$ as claimed.
\end{proof}


The irreducible $\mathbb{Z}$ modules are precisely $\mathbb{Z}/p \mathbb{Z}$ for $p$ prime. $R$ modules are just abelian groups in $\mathbb{Z}$ and irreducible modules are just those which have each element a generator by the above proof. Thus, by the fundamental theorem of cyclic groups each y $\mathbb{Z}/p \mathbb{Z}$ for $p$ prime is a irreducible module.

\subsection{Problem 18}
\subsubsection{Question}
Let $R$ be a Principal Ideal Domain and let $M$ be an $R$-module that is annihilated by the nonzero, proper ideal $(a)$. Let $a=p_1^{\alpha_1} p_2^{\alpha_2} \cdots p_k^{\alpha_k}$ be the unique factorization of $a$ into distinct prime powers in $R$. Let $M_i$ be the annihilator of $p_i^{\alpha_i}$ in $M$, i.e., $M_i$ is the set of $\{m\in M \ |\ p_i^{\alpha_i} m = 0 \}$ --- called the \emph{$p_i$-primary component of $M$}. Prove that 
\[M=M_1 \oplus M_2 \oplus \cdots \oplus M_k.\]
\subsubsection{Answer}
We will employ results from 16 and 17 without proof.

\begin{proof}
Since $(a)M = 0$ we know that $M/(a)M = M$. Moreover the ideals $p_i^{\alpha_i}$ are each pairwise comaximal so we can employ the result of Exercise 17 to conclude that
\[ M \cong M/(p_1^{\alpha_1})M \times \cdots \times M/(p_k^{\alpha_k})M. \]
Now we shall let $m = (m_1,\dots, m_k)$ be an element of this cartesian product. Suppose that $p_i^{\alpha_i} m = 0$. We then have $p_i^{\alpha_i} m_j = 0$ for all $j$. If $j \neq i$ then $p_j^{\alpha_j}m_j=0$ as well since $(p_i^{\alpha_i})$ and $(p_j^{\alpha_j})$ are comaximal this implies that $m_j=0$. Hence the $p_i^{\alpha_i}$-primary component of the cartesian product is $M/(p_i^{\alpha_i} M)$.

The isomorphism gives a direct product -- and therefore direct sum since we have a finite number of factors -- decomposition of $M$ into its $p_i^{\alpha_i}$-primary components. By definition each such component is just $M_i$. Hence 
\[M=M_1 \oplus M_2 \oplus \cdots \oplus M_k\]
as claimed.
\end{proof}


\subsection{Problem 20}
\subsubsection{Question}
Let $I$ be a nonempty index set and for each $i \in I $ let $M_i$ be an $R$-module. The \emph{direct product} of the modules $M_i$ is defined to be their direct product as abelian groups (cf. Exercise 15 in section 5.1) with the action of $R$ componentwise multiplication. The \emph{direct sum} of modules $M_i$ is defined to be the restricted direct product of the abelian groups $M_i$ (cf. Exercise 17 in Section 5.1) with the action or $R$ componentwise multiplication. In other words, the direct sum of the $M_i$'s is the subset of the direct product, $\prod_{i \in I} M_i$, which consists of elements $\prod_{i \in I}m_i$ such that only finitely many of the components  $m_i$ are nonzero; the action of $R$ on the direct product or direct sum is given by $r \prod_{i \in I} m_i = \prod_{i \in I}r m_i$ (cf. Appendix I for the definition of Cartesian products of infinitely many sets). The direct sum will  be denoted by $\oplus_{i \in I} M_i$.
\begin{enumerate}
\item Prove that the direct product of the $M_i$'s is an $R$-module and the direct sum of the $M_i$'s is a submodule of their direct product.
\item Show that if $R = \mathbb{Z}$, $I = \mathbb{Z}^+$ and $M_i$ is the cyclic group of order $i$ for each $i$, then the direct sum of the $M_i$'s is not isomorphic to their direct product. [Look at torsion.]
\end{enumerate}
\subsubsection{Answer}
\begin{enumerate}
\item First I prove that the direct product of the $M_i$'s is an $R$-module.

\begin{proof}
We have proven previously that the direct product of the abelian groups are abelian groups under componentwise addition. Thus in particular the direct product of each $M_i$ is an abelian group under componentwise addition.

The first property we must verify is that $(r+s)m = rm +sm$. This just follows from the same property in each $M_i$ though. In particular note that if $m_i$ is a component of some $m$ we have $(r+s)m_i = rm_i +sm_i$. Hence since our operations are defined componentwise we have $(r+s)m = rm +sm$ as desired.

Now we verify that $(rs) m = r (sm)$. Again, this fact follows from the corresponding property in each $M_i$. In particular note that if $m_i$ is a component of some $m$ we have $(rs) m_i = r (sm_i)$. Hence since our operations are defined componentwise we have $(rs) m = r (sm)$ as desired.

Now we show that $r(m+n) = rm +rn$ for any $m, n$ in the direct product. Just as in the previous two properties this just follows from the corresponding property in each $M_i$. In particular note that if $m_i, n_i$  are corresponding  components of some $m,n$ we have $r(m_i+n_i) = rm_i +rn_i$. Hence since our operations are defined componentwise we have $r(m+n) = rm +rn$.

Finally we note that if $1 \in R$ then $1m = m$ for all $m$ in the direct product since for each $m \in M_i$ we have $1m=m$.
\end{proof}

Now I will show that the direct sum is a submodule of the direct product using the submodule criterion.

\begin{proof}
It is readily apparent that the direct sum is a nonempty subset. Now we wish to show that for arbitrary $m, n$ in the direct sum and $r \in R$ we have $m+r n $ in the direct sum. 

By closure of each $M_i$ we know that $r n $ is in the direct sum. Any member of the direct sum has a finite number of nonzero components. Say that $m$ has $l$ and $n$ has $k$. Then $m+r n $ has at most $l +k$ nonzero components. So we have demonstrated the submodule criterion and shown that the direct sum is a submodule of the direct product as desired.
\end{proof}

\item If $A \cong B$ are two modules over the same ring $R$ and $A = \mathrm{Tor}(A)$ then $B = \mathrm{Tor}(B)$. 
\begin{proof}
Let $a \in A$. There exists some nonzero $r \in R$ such that $ra=0$. If $\varphi$ is the isomorphism then $\varphi(ra)=r\varphi(a)=0$. Since isomorphisms are bijective there exists some $a \in A$ such that $\varphi(a)=b$ for each $b \in B$. Hence there exists a nonzero $ r \in R $ such that $r b = 0$ for every $b \in B$ and $B = \mathrm{Tor}(B)$ by definition.
\end{proof}

Now I claim that the direct sum is a torsion module but that the direct product is not. For a proof of the first claim see \ref{done}
\begin{proof}
There exists no $r \in \mathbb{Z}$ such that $r(1,1,\dots)=(0,0,\dots)$. In particular assume that there is such an $r$. But $r 1 = r \neq 0 \in \mathbb{Z}/(r+1)\mathbb{Z}$. Contradiction. Hence the direct product is not a torsion module as claimed.
\end{proof}

\end{enumerate}


\subsection{Problem 22a}
\subsubsection{Question}
Let $R$ be a Principal Ideal Domain, let $M$ be a torsion $R$-module (cf. Exercise 4) and let $p$ be a prime in $R$ (do not assume $M$ is finitely generated, hence it need not have a nonzero annihilator --- cf. Exercise 5). The \emph{p-primary component } of $M$ is the set of all elements of $M$ that are annihilated by some positive power of $p$.
\begin{enumerate}
\item Prove that the $p$-primary component is a submodule. [See Exercise 13 in Section 1.]
\end{enumerate}
\subsubsection{Answer}
\begin{proof}
We will use the submodule criterion. 

The $p$-primary component is nonempty because in particular it contains 0. (0 is annihilated by any $r \in R$)

I claim that given $m, n$ in the $p$-primary component $r \in R$, $m +r n$ is in the $p$-primary component. Since $m, n$ in the $p$-primary component there are positive powers of $p$ which annihilate them. Say $p^\alpha m = p^\beta n = 0$, then  $p^{\alpha+\beta}(m +r n)=p^\beta (p^\alpha m) + r p^\beta (p^{\alpha}n )= 0+0=0$. Thus, the $p$-primary component is a submodule as claimed.
\end{proof}






\end{document}
