\documentclass[10pt]{article}
\setlength\headheight{14.5pt}
\title{Homework}
\author{Frederick Robinson}
\date{14 April 2010}
\usepackage{amsfonts}
\usepackage{fancyhdr}
\usepackage{amsthm}
\usepackage{setspace}
%\doublespacing
\pagestyle{fancyplain}

\begin{document}

\lhead{Frederick Robinson}
\rhead{Math 331: Algebra}

   \maketitle

\setcounter{tocdepth}{2} 

\tableofcontents

\section{Chapter 13 Section 2}
\subsection{Problem 13}
\subsubsection{Question}
Suppose $F=\mathbb{Q}(\alpha_1,\alpha_2,\dots,\alpha_n)$ where $\alpha_i^2\in \mathbb{Q}$ for $i=1,2,\dots, n$. Prove that $\sqrt[3]2\notin F$.
\subsubsection{Answer}
\begin{proof}
Since we have where $\alpha_i^2\in \mathbb{Q}$ for $i=1,2,\dots, n$ either $[\mathbb{Q}(\alpha_i): \mathbb{Q}]=1$ or  $[\mathbb{Q}(\alpha_i): \mathbb{Q}]=2$ for each $i$. Hence, $[F: \mathbb(Q)] = 2^k$ for some $k\neq 0$. Since the minimal polynomial over $\mathbb{Q}$ for $\sqrt[3]2$ is $x^3-2$ $[\mathbb{Q}(\sqrt[3]2:\mathbb{Q})]=3$. Hence, by Corollary 15, $\sqrt[3]2\notin F$ since $3$ does not divide $2^k$ for any $k \neq 0$
\end{proof}

\subsection{Problem 14}
\subsubsection{Question}
Prove that if $[F(\alpha):F]$ is odd then $F(\alpha)=F(\alpha^2)$.
\subsubsection{Answer}
\begin{proof}
It is clear by closure that $F(\alpha^2) \subset F(\alpha)$. Moreover, since $\alpha \in F(\alpha)$ we have $[F(\alpha): F(\alpha^2)]=1$ or $[F(\alpha): F(\alpha^2)]=2$. Thus, by multiplicativity of extension degree, together with the fact that $[F(\alpha):F]$ is odd  $[F(\alpha): F(\alpha^2)]=1$ and $F(\alpha) \subset F(\alpha^2) \Rightarrow F(\alpha)=F(\alpha^2)$.
\end{proof}

\subsection{Problem 16}
\subsubsection{Question}
Let $K/F$ be an algebraic extension and let $R$ be a \emph{ring} contained in $K$ and containing $F$. Show that $R$ is a subfield of $K$ containing $F$.
\subsubsection{Answer}
\begin{proof}
Since given $x \in R$ we have also $x \in K$, we know that any $x \in R$ commutes with any $y \in R$. As $1 \in F$ we know further that $1 \in R$. Finally, any $x \in K $ is the root of some nonzero polynomial with coefficients in $F$. Say in particular,
\[0 = a_0 +x a_1 + \cdots + x^n a_n\] 
for $a_i \in F$. Hence,
\begin{eqnarray*}-a_0 = x a_1 + \cdots + x^n a_n  &\Rightarrow& -a_0 = x(  a_1 + \cdots + x^{n-1} a_n) \\ &\Rightarrow& 1 =  x \left( ( a_1 + \cdots + x^{n-1} a_n) ( -a_0)^{-1}\right) \end{eqnarray*}
So, every $x \in R$ has an inverse. Thus, $R$ is a field as desired.
\end{proof}

\subsection{Problem 19}
\subsubsection{Question}
Let $K$ be an extension of $F$ of degree $n$. 
\begin{enumerate}
\item For any $\alpha \in K$ prove that $\alpha$ acting by left multiplication on $K$ is an $F$-linear transformation of $K$.
\item Prove that $K$ is isomorphic to a subfield of the ring of $n \times n$ matrices over $F$, so the ring of $n \times n$ matrices over $F$ contains an isomorphic copy of \emph{every} extension of $F$ of degree $\leq n$.
\end{enumerate}
\subsubsection{Answer}
\begin{enumerate}
\item \label{first}
\begin{proof}
Let $\{ 1, \theta_1, \dots, \theta_{n-1}\}$ be a basis for $K$ viewed as a vector space over $F$. Furthermore let $x, y \in K$ with $x=x_0+x_1 \theta_1 +\cdots + x_{n-1}\theta_{n-1}, y=y_0+ y_1\theta_1 + \cdots y_{n-1} \theta_{n-1} $ ($x_i,y_i \in F$) and let $c \in F$.
\begin{eqnarray*}f(x+y)&=&\alpha (x_0+x_1 \theta_1 +\cdots + x_{n-1}\theta_{n-1}+y_0+ y_1\theta_1 + \cdots y_{n-1} \theta_{n-1})\\ 
&=&\alpha (x_0+x_1 \theta_1 +\cdots + x_{n-1}\theta_{n-1})+\alpha(y_0+ y_1\theta_1 + \cdots y_{n-1} \theta_{n-1})\\
&=&f(x)+f(y)
\end{eqnarray*}
Now check
\begin{eqnarray*}
f(c x) &=& \alpha  (cx_0+cx_1 \theta_1 +\cdots + cx_{n-1}\theta_{n-1})\\
&=& c \alpha (x_0+x_1 \theta_1 +\cdots + x_{n-1}\theta_{n-1})\\
&=& c f (x_0+x_1 \theta_1 +\cdots + x_{n-1}\theta_{n-1})\\
&=& c f (x)\\
\end{eqnarray*}\end{proof}
\item \begin{proof}Let
\[\varphi : K \to M_n(F)\]
be the mapping which takes an element of $K$ to the linear transformation given by left multiplication by that element. Part \ref{first} establishes that this mapping is well defined. 

Given $x, y \in K$ we have $\varphi(x \cdot y)=\varphi(x)\varphi(y)$ and  $\varphi(x+y) = \varphi(x) +\varphi(y)$
so $\varphi$ is indeed a homomorphism of rings. Moreover, $\varphi$ is injective since $\varphi(x)=\varphi(y) \Rightarrow x z = y z $ for all $z \in K$. In particular take $z =1 \Rightarrow x=y$.

By definition $\varphi|_{Im{\varphi}}$ (the restriction of the codomain of $\varphi$ to its image) is surjective. Finally note that the image of $\varphi$ is a field since it is isomorphic to $K$ (a field) so the image of $\varphi$ fulfills the required conditions.
\end{proof}
\end{enumerate}

\subsection{Problem 20}
\subsubsection{Question}
Show that if the matrix of the linear transformation ``multiplication by $\alpha$" considered in the previous exercise is $A$ then $\alpha$ is a root of the characteristic polynomial for $A$. This gives an effective procedure for determining an equation of degree $n$ satisfied by an element $\alpha$ in an extension of $F$ of degree $n$. Use this procedure to obtain the monic polynomial of degree $3$ satisfied by $\sqrt[3]2$ and by $1+\sqrt[3]2+\sqrt[3]4$.
\subsubsection{Answer}
\begin{proof}
Recall that $x \in K$ is a root of the characteristic polynomial if and only if $x$ is an eigenvalue. However, $\alpha$ is an eigenvalue corresponding to the eigenvector $\{1,\theta_1, \dots, \theta_{n-1}\}$ (where the $\theta_i$ form the basis for $K$ as a vector space over $F$) by construction. Thus,  $\alpha$ is a root of the characteristic polynomial, as desired.
\end{proof}
Now, consider the basis given by $\{1,\sqrt[3]2,\sqrt[3]4\}$. We can compute
\[\varphi(\sqrt[3]2)=\left[ \begin{array}{ccc} 0&1&0 \\ 0&0&1\\2&0&0 \end{array} \right] \]
since 
\[1 \sqrt[3]2 = \sqrt[3]2, \sqrt[3]2 \sqrt[3]2 = \sqrt[3]4 ,\mathrm{and \ } \sqrt[3]4 \sqrt[3]2 = 2.\] The characteristic polynomial of this matrix is $2-x^3$ which has $\sqrt[3]2$ as a root. Similarly we compute
\[\varphi(1+\sqrt[3]2+\sqrt[3]4)=\left[ \begin{array}{ccc}1&1&1\\2&1&1\\2&2&1 \end{array}\right]\]
since 
\[ 1(1+\sqrt[3]2+\sqrt[3]4) = 1+\sqrt[3]2+\sqrt[3]4, \sqrt[3]2 (1+\sqrt[3]2+\sqrt[3]4) = 2 + \sqrt[3]2 + \sqrt[3]4\]
\[\mathrm{ and\ } \sqrt[3]4(1+\sqrt[3]2+\sqrt[3]4) = 2 + 2 \sqrt[3]2 + \sqrt[3]4\]
The characteristic polynomial of this matrix is $1+3 x+3 x^2-x^3$ which has $1+\sqrt[3]2+\sqrt[3]4$ as a root, as desired.
\section{Chapter 13 Section 3}
\subsection{Problem 1}
\subsubsection{Question}
Prove that it is impossible to construct the regular $9$-gon.
\subsubsection{Answer}
\begin{proof}
If we can construct the regular $9$-gon we can construct the angle $2 \pi / 9$ from it by raising bisectors of two adjacent interior angles and examining the angle formed by their intersection.

The angle $2 \pi / 9 = 40^o$ is not constructible for, we know from elementary geometry that we can bisect an angle and the angle $\pi / 9$ is not constructible by Theorem 24 (Page 534). 
\end{proof}

\subsection{Problem 4}
\subsubsection{Question}
The construction of the regular $7$-gon amounts to the constructibility of $\cos{(2 \pi /7)}$. We shall see later (Section 14.5 and Exercise 2 of Section 14.7) that $\alpha = 2 \cos{(2 \pi /7)}$ satisfies the equation $x^3+x^2-2x -1=0$. Use this to prove that the regular $7$-gon is not constructible by straightedge and compass.
\subsubsection{Answer}
\begin{proof}
Assume towards a contradiction that the regular $7$-gon is constructible by straightedge and compass. Then the $\cos{(2 \pi / 7)}$ (and by closure $\alpha = 2 \cos{(2 \pi /6)}$)  is constructible. The minimal polynomial for $\alpha$ over $\mathbb{Q}$ is $f(x)=x^3+x^2-2x-1$ since $\alpha$ is a root of $f$, $f$ is monic, and $f$ is irreducible. 

To verify this irreducibility note that $f(x)$ is reducible if and only if $f(x+2)$ is irreducible. However $f(x+2)=7+14 x+7 x^2+x^3$ which is irreducible by Eisenstein. Thus $[\mathbb{Q}(\alpha): \mathbb{Q}]=3 \neq 2^k$ for any $k$, a contradiction.
\end{proof}

\subsection{Problem 5}
\subsubsection{Question}
Use the fact that $\alpha =2 \cos{(2\pi /5)}$ satisfies the equation $x^2+x-1=0$ to conclude that the regular $5$-gon is constructible by straightedge and compass.
\subsubsection{Answer}
\begin{proof}
Since $\alpha$ is a root of a polynomial of degree 2 we can conclude that $\alpha$ has either $[\mathbb{Q}(\alpha): \mathbb{Q}]=2$ or $[\mathbb{Q}(\alpha): \mathbb{Q}]=1$. In either case $\alpha$, and by closure $\alpha / 2$ is constructible.   

Since $\cos{(2 \pi / 5)}$ is constructible so is the angle $2 \pi / 5$. In particular construct a `base' segment of length $\cos{(2 \pi /5)}$ and raise a perpendicular from one end, then draw a circle of radius 1 centered at the other end of our segment. The triangle formed by taking as vertices both ends of the original segment, together with an intersection of our circle with the perpendicular contains an angle of $2 \pi / 5$.

After having performed this construction once we are left with a triangle one of whose sides has unit length. Fix this side, and mark off from it a new segment of length $\cos{(2 \pi / 5)}$ one of whose ends is the vertex of the triangle adjacent to the $2 \pi / 5$ angle. Now, we can reproduce our original construction with this as the base (making sure to construct the triangle whose $2 \pi / 5 $ angle does not coincide with the one previously constructed ).

Once we perform this construction a total of 5 times (really 4 probably suffice) we have 5 triangles each of which share a common vertex adjacent to the angle $2 \pi / 5$. If we fix a vertex in each of these triangles a unit from the shared vertex and connect them in the obvious way we will have formed the regular pentagon as desired.

It is worth noting that this construction will work in general for any regular $n$-gon, given that $\cos{(2 \pi / n)}$ is constructible.
\end{proof}

\end{document}
