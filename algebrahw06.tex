\documentclass[12pt]{article}
\setlength\headheight{14.5pt}
\title{Homework}
\author{Frederick Robinson}
\date{10 February 2010}
\usepackage{amsfonts}
\usepackage{fancyhdr}
\usepackage{amsthm}
\usepackage{setspace}
\doublespacing
\pagestyle{fancyplain}

\begin{document}

\lhead{Frederick Robinson}
\rhead{Math 331: Algebra}

   \maketitle

\setcounter{tocdepth}{2} 

\tableofcontents

\section{Chapter 7 Section 5}
\subsection{Problem 2}

\subsubsection{Question}
Let $R$ be an integral domain and let $D$ be a nonempty subset of $R$ that is closed under multiplication.  Prove that the ring of fractions $D^{-1}R$ is isomorphic to a subring of the quotient field of $R$ (hence is also an integral domain).
\subsubsection{Answer}
I claim that $D^{-1}R$ is isomorphic to a subring of the quotient field of $R$, in particular I claim that the subset of the quotient field of $R$ defined by $D^{-1}R$ is itself a subring.

I must therefore demonstrate several properties. Since $D$ is nonempty it contains some $r$ which is in $R$ and therefore $D^{-1}R$ contains the identity in the quotient field of $R$ (just $r^{-1}r=1$). Moreover, is closed under multiplication since we know that both $D^{-1}$ and $R$ are closed under multiplication. Additionally we have that $D^{-1}R$ is closed under addition since $D^{-1}$ is closed under multiplication (so we can take common denominators) and $R$ is closed under addition.

Since by demonstrating that $D^{-1}R$ contains identity we showed that it is nonempty too we have demonstrated that the subset of the quotient field of $R$ defined by $D^{-1}R$ is a subring as claimed.

\subsection{Problem 4}

\subsubsection{Question}
Prove any subfield of $\mathbb{R}$ must contain $\mathbb{Q}$.
\subsubsection{Answer}
Let $S$ be a subfield of $\mathbb{R}$. $S$ must contain $1\neq 0 $ since it is a subfield. Moreover in order to be closed under addition it must contain each $n \in \mathbb{N}$ since we can write for each such $n$, $n = \underbrace{1+\cdots+1}_{n\mathrm{\ times}}$. Since $S$ is a field, and must therefore be closed under the taking of inverses, it must also contain each inverse, that is $1/n$ for each $n \in \mathbb{N}$. Finally, it must contain each $p/q \in \mathbb{Q}$ by closure under multiplication since such can be expressed in the form $(1/q) \cdot p$. Thus, any subfield of $\mathbb{R}$ must contain $\mathbb{Q}$ as claimed.
\section{Chapter 8 Section 1}
\subsection{Problem 3}

\subsubsection{Question}
Let $R$ be a Euclidean Domain. Let $m$ be the minimum integer in the set of norms of nonzero elements of $R$. Prove that every nonzero element of $R$ of norm $m$ is a unit. Deduce that a nonzero element of norm zero (if such an element exists) is a unit.
\subsubsection{Answer}
Let $x \in R$ such that $N(x)=m$. By the division algorithm $1 = q x + r$ for some $q,r \in R$ with $r=0$ or $N(r) < N(b)$. Since $N(m)$ is minimal it must be that $r = 0$. So, $1=q x$ and $x$ is a unit. So, a nonzero element of norm zero must also be a unit since such an element (if it exists) has minimum norm.


\subsection{Problem 10}

\subsubsection{Question}
Prove that the quotient ring $\mathbb{Z}[i]/I$ is finite for any nonzero ideal $I$ of $\mathbb{Z}[i]$. [Use the fact that $I=(\alpha)$ for some nonzero $\alpha$ and then use the Division Algorithm in this Euclidean Domain to see that every coset of $I$ is represented by an element of norm less than $N(\alpha)$.]
\subsubsection{Answer}
Let $I$ be a nonzero ideal of $\mathbb{Z}[i]$. Since $\mathbb{Z}[i]$ is a Principal Ideal Domain we can write $I = (\alpha)$ for some $\alpha \neq 0$. Moreover, since $\mathbb{Z}[i]$ is a Euclidean Domain, given an arbitrary $x \in \mathbb{Z}[i]$ we can write $x = \alpha q + r$ for some $r$ with $N(r)< N(\alpha)$ or $r=0$. Since there are but a finite number of such $r$ there are but a finite number of cosets of $(\alpha)$ in $\mathbb{Z}[i]$ as claimed.


\section{Chapter 8 Section 2}
\subsection{Problem 3}

\subsubsection{Question}
Prove that a quotient of a P.I.D. by a prime ideal is again a P.I.D.
\subsubsection{Answer}
Let $P$ be a Principal Ideal Domain, and $Q$ be a prime ideal in $P$. We seek to prove that $P/Q$ is a Principal Ideal Domain. If $Q$ is the zero ideal then $P/Q$ must be a Principal Ideal Domain since it is just isomorphic to $P$ which is itself a Principal Ideal Domain.

If however, $Q$ is nonzero then $Q$ is maximal in $P$ by Proposition 7. Let $L$ be an ideal in $P/Q$. Then the preimage $M=\varphi^{-1}(L)$ of $L$ under the natural ring homomorphism $\varphi:P \to P/Q$ is an ideal in $P$. Then since $Q$ is a maximal ideal in $P$ a P.I.D. there are two cases. Either $M \subseteq  Q$  in which case $L = (1)$ or $P = Q M$ (if neither of these is the case then the $Q$ is not maximal as it is contained in $Q M $). If $P = Q M$ then each $p \in P$ can be written as $q m = p$ for some $q \in Q$ and $m \in M$. Applying the homomorphism property this implies that $\varphi(q) \varphi(m) = \varphi(p)$ and  since each element of $P/Q$ can be written as $\varphi(p)$ for some $p$ and $\varphi(q)=1$ it must be that $\varphi(M)=L=P/Q$. So in this case too, $L$ is Principal as in particular $(\overline{1})=L$.

So we have established that a quotient of a P.I.D. by a prime ideal is again a P.I.D as desired.

\subsection{Problem 4}

\subsubsection{Question}
Let $R$ be an integral domain. Prove that if the following two conditions hold then $R$ is a Principle Ideal Domain.
\begin{enumerate}
\item \label{first} any two nonzero elements $a$ and $b$ in $R$ have a greatest common divisor which can be written in the form $r a + s b $ for some $r,s \in R$, and
\item \label{second} if $a_1,a_2,a_3,\dots$ are nonzero elements of $R$ such that $a_{i+1}|a_i$ for all $i$, then there is a positive integer $N$ such that $a_n$ is a unit times $a_N$ for all $n \geq N$.
\end{enumerate}
\subsubsection{Answer}
Suppose that $I \subset R$ is an ideal in $R$. Call its elements $I=\{i_1,i_2,\dots\}$. Then construct  a sequence $G=\{g_1,g_2,\dots\}$ from $I$ in the following manner. 
\[g_j=\left\{ \begin{array} {lr} \gcd{(i_1,i_2)} & j=1\\ \gcd{(i_{j+1},g_{j-1})} &\mathrm{otherwise} \end{array} \right.\]
By Property \ref{first} this sequence $G$ must exist, and be a subset of the ideal $I$ since an element of the form $r a + s b$ for $a, b \in I$ must also be in $I$.  Moreover, the element $g_N$ whose existence is ensured by Property \ref{second} must generate the ideal $I$. It divides every element $i_n$ for $n<N$ since it divides a divisor of each such element (namely $g_n$). It also divides every $i_n$ with $n > N$ since it is an associate of a divisor of each such $i_n$ by Property \ref{second}.

So we have demonstrated a generator for an arbitrary ideal $I$ in $R$. Hence, $I$ is a Principal Ideal Domain.


\subsection{Problem 5}

\subsubsection{Question}
Let $R$ be the quadratic integer ring $\mathbb{Z}[\sqrt{-5}]$. Define the ideals $I_2 = (2,1+\sqrt{-5}), I_3=(3,2+\sqrt{-5}),$ and $I'_3= (3,2-\sqrt{-5})$.
\begin{enumerate}
\item Prove that $I_2$, $I_3$, and $I'_3$ are nonprincipal ideals in $R$. [Note that Example 2 following Proposition 1 proves this for $I_3$.]
\item Prove that the product of two nonprincipal ideals can be principal by showing that $I_2^2=(2)$.
\item Prove similarly that $I_2 I_3 = (1-\sqrt{-5})$ and $I_2 I'_3= (1+\sqrt{-5})$ are principal. Conclude that the principal ideal (6) is the product of 4 ideals: $(6) =I_2^2 I_3 I'_3$.
\end{enumerate}
\subsubsection{Answer}
\begin{enumerate}
\item As this statement is proven in the book for for $I_3$ we will prove it only for $I_2$ and $I'_3$. 

Let $N$ be the norm associated with the quadratic integer ring $\mathbb{Z}[\sqrt{-5}]$ defined by $N(a+b\sqrt{-5})=a^2+5b^2$. Suppose that $I_2$ were principal. Then there would be some $a + b\sqrt{-5}$ so that $a + b\sqrt{-5}$ generates $I_2$. In particular we would have $2=\alpha (a+ b \sqrt{-5})$ and $1+\sqrt{-5}=\beta(a+b\sqrt{-5})$ for some $\alpha, \beta \in \mathbb{Z}[\sqrt{-5}]$. 

Norming each of these we see that we would need $4= N(\alpha)(a^2+5b^2)$ and $6= N(\beta)(a^2+5b^2)$. However this would necessitate by the first equation $b=0$, and $a=1$ or $a=2$. Taking the second into account we would need $a=1$ since there is no integer $N(\beta)$ that would give $N(\beta) \cdot 4 = 6$. However, if $a=1$ and $b=0$ this would just be the generator for the entire ring, and $I_2$ is not the entire ring since it does not contain $1$.

Similarly for $I'_3$ assuming the existence of a generator $a+b\sqrt{-5}$ implies that $9=N(\alpha)(a^2+5b^2)$ and $9=N(\beta)(a^2+5b^2)$. Then, it is easy to check that the only options for generators are the ones which are listed already. However, now we can see that there can be no single generator for the entire ideal as there are elements of $I'_3$ which are generated by one of the generators but not the other. In particular $3 \notin (2-\sqrt{-5})$ and $3-\sqrt{-5} \notin (3)$.
\item Each element of $I_2^2 $ is of the form $a(2)b(1+\sqrt{-5})$ or $a(2)b(2)$ or $a(1+\sqrt{-5})b(1+\sqrt{-5})$.  It should be clear that the elements of the first forms are members of $(2)$. We check elements of the last form. Expanding we see that they can be written as $-14(a b)$ and are thus members of $(2)$. Hence each element of $I_2^2$ is a member of $(2)$ as claimed. Since each element of $(2)$ is in the squared ring.
\item Elements of $I_2 I_3$ can be written as products of the form $a(3)(2)$ or $(2)(2+\sqrt{-5})(a)$ or $(1+\sqrt{-5})(3)(a)$ or $a(1+\sqrt{-5})(2+\sqrt{-5})$. Since these forms can be expanded as $(6)a$, $(4+2\sqrt{-5})a$, $(3+3\sqrt{-5})a$, and $(-3+3\sqrt{-5})a$ respectively $I_2 I_3 = (1+\sqrt{-5})$ as claimed. We can show a simlar statement for $I_2 I'_3$ and so it follows that $(6) = I_2^2 I_3 I'_3$
\end{enumerate}

\end{document}
