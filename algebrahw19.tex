\documentclass[10pt]{article}
\setlength\headheight{14.5pt}
\title{Homework}
\author{Frederick Robinson}
\date{28 May 2010}
\usepackage{amsfonts}
\usepackage{mathabx}
\usepackage{fancyhdr}
\usepackage{amsthm}
\usepackage{setspace}
%\doublespacing
\pagestyle{fancyplain}

\begin{document}

\lhead{Frederick Robinson}
\rhead{Math 331: Algebra}

   \maketitle

\setcounter{tocdepth}{2} 

%\tableofcontents

\section{Chapter 14 Section 6}
\subsection{Problem 3}
\subsubsection{Question}
Prove that for any $a, b \in \mathbb{F}_{p^n}$ that if $x^3+ax+b$ is irreducible then $-4a^3-27b^2$ is a square in $\mathbb{F}_{p^n}$.
\subsubsection{Answer}
We have from the book that the discriminant for a cubic is
\[D = a^2 b^2 -4b^3-4a^3 c-27c^2 + 18 abc.\]
However, in the notation of the book we have $a=0$, $b=a$, $c=b$. Substituting reveals that in our case
\[D = -4a^3 -27 b^2.\]

Since the Galois group corresponding to a finite extension of a finite field is cyclic we know that the Galois group in this case is the Cyclic group of order 3, and that the discriminant must be a square in the base field.

\subsection{Problem 11}
\subsubsection{Question}
Let $F$ be an extension of $\mathbb{Q}$ of degree 4 that is not Galois over $\mathbb{Q}$. Prove that the Galois closure of $F$ has Galois group either $S_4$, $A_4$ or the dihedral group $D_8$ of order 8. Prove that the Galois group is dihedral if and only if $F$ contains a quadratic extension of $\mathbb{Q}$
\subsubsection{Answer}

Say that $E /  \mathbb Q = \bar{F}$. For some root $\alpha \in F$ we can say $F= \mathbb{Q}(\alpha)$, so $E$ is the splitting field of teh minimal polynomial of $\alpha$. Since this is a degree 4 polynomial we know in particular that $G= \mathrm{Gal}(E / \mathbb{Q})$ is a subgroup of $S_4$. 

Since $E$ has a subfield that is 4th degree in $\mathbb{Q}$, $G$ has a subgroup of index $4$. We know that $|G|>4$ since $F$ is not Galois over $\mathbb{Q}$. Therefore $|G|=8,12,$ or $24$. 

In the case that $|G|= 8$ we have $G=D_8$, the only group of order 8 that has a subgroup which is not normal and therefore corresponds to $F$. In the case that $|G|=24$, $G$ is just $S_4$ itself. If $|G|=12$ it is just the only index 2 subgroup of $S_4$, $A_4$.

$F$ contains a quadratic extension of $\mathbb{Q}$ if and only if each index 4 subgroup of $G$ is contained in an index 2 subgroup. Both $S_4$, and $A_4$ fail this condition, but each element of $D_8$ having order 2 is contained in a subgroup of order 4.

\subsection{Problem 17}
\subsubsection{Question}
Find the Galois group of $x^4-7$ over $\mathbb{Q}$ explicitly as a permutation group on the roots.
\subsubsection{Answer}
Denote the roots of $f(x) = x^4-7 $ by
\[ \pm \alpha = \pm \sqrt[4] 7  \quad \pm \beta= \pm i \sqrt[4]7. \]
It is easy to verify that the Klein 4-group generated by 
\[\sigma = (-\alpha\  \alpha)\quad \tau = (-\beta\ \beta)\]
is a subgroup of the Galois group. Moreover, since the resolvent cubic is $h(x)=x^3+28$ which splits into a linear and a quadratic term this is the entire Galois group.

\subsection{Problem 19}
\subsubsection{Question}
Let $f(x)$ be an irreducible polynomial of degree 4 in $\mathbb{Q}[x]$ with discriminant $D$. Let $K$ denote the splitting field of $f(x)$, viewed as a subfield of the complex numbers $\mathbb{C}$.
\begin{enumerate}
\item Prove that $\mathbb{Q}(\sqrt D) \subset K$.
\item Let $\tau$ denote complex conjugation and let $\tau_K$ denote the restriction of complex conjugation to $K$. Prove that $\tau_K$ is an element of $\mathrm{Gal}(K/\mathbb{Q})$ of order 1 or 2 depending on whether every element of $K$ is real or not.
\item Prove that if $D<0$ then $K$ cannot be cyclic of degree 4 over $\mathbb{Q}$ (i.e., $\mathrm{Gal}(K/\mathbb{Q})$ cannot be a cyclic group of order 4).
\item Prove generally that $\mathbb{Q}(\sqrt D )$ for squarefree $D <0$ is not a subfield of a cyclic quartic field (cf. also Exercise 19 of Section 7).
\end{enumerate}
\subsubsection{Answer}
\begin{enumerate}
\item \begin{proof}
\[\sqrt D = \prod_{i <j} (x_i - x_j)\]
for $x_i$ a root of $f$. However every root of $F$ is in $K$. Hence $\sqrt D \in K \Rightarrow \mathbb{Q}(\sqrt D) \subset K$ as claimed. \end{proof}
\item \begin{proof} If $K$ is real then clearly $\tau_K$ is just the identity permutation, and thus of order 1. In the case that there is some root of $K$ which is not real $\tau_K$ is a member of the Galois group since complex conjugation is an automorphism of $\mathbb{C}$ which is surjective in the restriction (as roots are in conjugate pairs). Moreover it must be of order 2 since the unresticted automorphism has this property and there is, by assumption at least one element of $K$ which is not taken by complex conjugation to itself.\end{proof}
\item \begin{proof}Suppose towards a contradiction that $D<0$ with $K$ cyclic of order 4. Then, since $K$ is cyclic and $f$ is irreducible $K = F(\alpha)$ for any root $\alpha$ of $f$. 

Since $D<0$  $f$ cannot have a real root. Thus, all roots have nonzero imaginary part. Since roots occur in complex conjugate pairs we may compute 
\begin{eqnarray*}
\sqrt D &=& (x - \bar x)( x - y)( x - \bar y)(\bar x - y)( \bar x -\bar y)(y - \bar y)\\
&=& (x - \bar x)(y - \bar y) (x- y ) \overline{(x- y)} (x - \bar y) \overline{(x-\bar y)} \\
&=& 4 \Re x\Re y (x- y ) \overline{(x- y)} (x - \bar y) \overline{(x-\bar y)}\\
&=& 4 \Re x\Re y \left| x- y \right|^2\left| x - \bar y\right|^2
\end{eqnarray*}
but this is real,  contradicting our assumption that $D<0.$
\end{proof}
\item A degree 4 cyclic group has only one proper, nontrivial subgroup. That is the index 2 subgroup. By the fundamental theorem then there is only one quadratic extension of $\mathbb{Q}$ in a cyclic quartic field $K$. The only possibility is that this is $\mathbb{Q}(\sqrt D) $. So, we conclude that if $\mathbb{Q}(\sqrt E) \subset K$ for $E$ square free we have $E=D$ (by the previous part $D \geq 0$)
\end{enumerate}

\subsection{Problem 20}
\subsubsection{Question}
Determine the Galois group of $(x^3-2)(x^3-3)$ over $\mathbb{Q}$. Determine all the subfields which contain $\mathbb{Q}(\rho)$ where $\rho$ is a primitive $3^\mathrm{rd}$ root of unity.
\subsubsection{Answer}
The Galois group restricted to just the roots of one of the irreducible factors must be a subgroup of the Galois group corresponding to that irreducible factors. 

Any member of the Galois group must therefore be expressible as a product of the following
\[ \sigma_2 : \left\{ \begin{array}{c}  \sqrt[3]2 \mapsto \rho \sqrt[3]2 \\ \rho \mapsto \rho \end{array} \right. \quad \tau_2:  \left\{ \begin{array}{c} \sqrt[3]2 \mapsto \sqrt[3]2 \\ \rho \mapsto \rho^2 \end{array} \right.  \]
\[ \sigma_3 : \left\{ \begin{array}{c}  \sqrt[3]3 \mapsto \rho \sqrt[3]3 \\ \rho \mapsto \rho \end{array} \right. \quad \tau_3:  \left\{ \begin{array}{c} \sqrt[3]3 \mapsto \sqrt[3]3 \\ \rho \mapsto \rho^2 \end{array} \right.  \]
and the Galois group is isomorphic to $\mathbb{Z}/3\mathbb{Z}\times\mathbb{Z}/3\mathbb{Z}\times\mathbb{Z}/2\mathbb{Z}$.

The subfields containing $\rho$ are the subfields fixed by automorphisms which fix $\rho$. There are nine such automorphisms. Namely the ones generated by $\sigma_2, \sigma_3$.
\subsection{Problem 33}
\subsubsection{Question}
\begin{enumerate}
\item Prove that the discriminant of the cyclotomic polynomial $\Phi_p(x)$ of the $p^\mathrm{th}$ roots of unity for an odd prime $p$ is $(-1)^{(p-1)/2}p^{p-2}$. [One approach: use Exercise 5 of the previous section together with the determinant form for the discriminant  in terms of the power sums $p_i$.]
\item Prove that $\mathbb{Q}(\sqrt{(-1)^{(p-1)/ 2}p})\subset \mathbb{Q}(\zeta_p)$ for $p$ an odd prime. (Cf. also Exercise 11 of Section 7.)
\end{enumerate}
\subsubsection{Answer}
\begin{proof}
We have from exercise 32 that 
\[D = (-1)^{n(n-1)/2}R(f,f')\]
given
\[R(f,f')=\prod_{i=1}^n f'(\alpha_i)\]
Since in this particular instance we know that
\[\Phi_p=x^{p-1}+ x^{p-2}+ \cdots +1\]
we have in particular
\[\Phi'_p = (p-1)x^{p-2}+(p-2)x^{p-3}+ \cdots + 1.\]
Evaluating the product we get
\[\prod_{i=1}^n f'(\alpha_i) = i^{-1+2 n-n^2} p^{p-2}\]
by employing the result of exercise 5 from the previous section.
\end{proof}

The second statement follows from part 1 of the previous exercise.
\end{document}
