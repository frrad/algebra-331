\documentclass[10pt]{article}
\setlength\headheight{14.5pt}
\title{Homework}
\author{Frederick Robinson}
\date{17 May 2010}
\usepackage{amsfonts}
\usepackage{fancyhdr}
\usepackage{amsthm}
\usepackage{setspace}
%\doublespacing
\pagestyle{fancyplain}

\begin{document}

\lhead{Frederick Robinson}
\rhead{Math 331: Algebra}

   \maketitle

\setcounter{tocdepth}{2} 

%\tableofcontents

\section{Chapter 14 Section 2}
\subsection{Problem 5}
\subsubsection{Question}
Prove that the Galois group of $x^p-2$ for $p$ a prime is isomorphic to the group of matrices $\left( \begin{array}{cc} a & b \\0&1\end{array}\right)$ where $a, b \in \mathbb{F}_p$, $ a \neq 0$.
\subsubsection{Answer}
By the previous exercise we know that if $\theta$ is the real $\sqrt[p]2$ and $\zeta$ a principle $p$th root of unity then all elements of the group are $\sigma_{(m,n)}$ ($0 \leq n \leq p-1$, $1 \leq m \leq p-1$ ) where
\[\sigma_{(m,n)} : \left\{ \begin{array}{ll} \zeta \mapsto \zeta^m \\ \theta \mapsto \theta\zeta^n  \end{array}\right.  \]

I claim that the correspondence between this group and the one provided defined by
\[ \varphi: \sigma_{(m,n)} \mapsto \left( \begin{array}{cc} m &n \\ 0&1 \end{array} \right)\]
is an isomorphism.
\begin{proof}
This correspondence is clearly bijective, so it suffices to show that it is a homomorphism. We compute
\[ \sigma_{(m_1,n_1)} \cdot \sigma_{(m_2,n_2)}  (\zeta)  = \zeta^{m_1 m_2}\]
\begin{eqnarray*} \sigma_{(m_1,n_1)} \cdot \sigma_{(m_2,n_2)}  (\theta)  &=& \sigma_{(m_1,n_1)}(\theta \zeta^{n_2}) \\ &=& \theta \zeta^{n_1} \zeta^{m_1 n_2} \\&=& \theta \zeta^{n_1+m_1n_2}.\end{eqnarray*}
Moreover, we can compute
\[\left( \begin{array}{cc} m_1&n_1\\0&1 \end{array}\right)  \cdot \left( \begin{array}{cc} m_2&n_2\\0&1 \end{array}\right) = \left( \begin{array}{cc} m_1 m_2 &n_1 +m_1 n_2\\0&1 \end{array}\right). \]
Thus, $\varphi(\sigma \tau) =\varphi(\sigma) \varphi(\tau)$ and $\varphi$ is a homomorphism as claimed.
\end{proof}

\subsection{Problem 13}
\subsubsection{Question}
Prove that if the Galois group of the splitting field of a cubic over $\mathbb{Q}$ is the cyclic group of order 3 then the roots of the cubic are real.
\subsubsection{Answer}
\begin{proof}
Let $f \in \mathbb{Q}[x]$ be a cubic and suppose that $f$ has a complex root. The Galois group has a subgroup generated by complex conjugation since it has a complex root. This subgroup is $\mathbb{Z}/2\mathbb{Z}$ which is not a subgroup of $\mathbb{Z}/3\mathbb{Z}$ so the Galois group of $f$ is not $\mathbb{Z}/3\mathbb{Z}$.
\end{proof}

\section{Chapter 14 Section 3}
\subsection{Problem 1}
\subsubsection{Question}
Factor $x^8-x$ into irreducibles in $\mathbb{Z}[x]$ and in $\mathbb{F}_2[x]$.
\subsubsection{Answer}
In $\mathbb{Z}[x]$ we can factor $f(x)=x^8-x$ as $ f(x)=x (x-1)\left(1+x+x^2+x^3+x^4+x^5 + x^6\right)$. The degree $6$ polynomial in this factorization is irreducible since $f(x+1)= 7+21 x+35 x^2+35 x^3+21 x^4+7 x^5+x^6$ is Eisenstein (with $p=7$).

By Proposition 18 we have in $\mathbb{F}_2$ that $f$ is the product of all the distinct irreducible polynomials in $\mathbb{F}_p[x]$ of degree $d$ where $d$ runs through all divisors of $3$. In particular this means that $f$ is the product of all distinct irreducible polynomials of degrees 1,3 in $\mathbb{F}_2$. Hence
\[f(x) = x(x-1)(x^3+x^2+1)(x^3+x+1) .\]
We can check that
\[ x(x-1)(x^3+x^2+1)(x^3+x+1) = x^8+2x^5-2x^4-x\]
which indeed reduces mod 2 to $x^8-x$ as claimed.


\subsection{Problem 3}
\subsubsection{Question}
Prove that an algebraically closed field must be infinite.
\subsubsection{Answer}
\begin{proof}
Fix some finite field. By Proposition 15 this field is just $\mathbb{F}_{p^n}$ for some prime $p$, integer $n \geq 1$. However, from the book (page 588) we have
\[\overline{\mathbb{F}_p} = \bigcup_{n \geq1} \mathbb{F}_{p^n}\]
and so we may conclude that the algebraic closure of $\mathbb{F}_p$, a subfield of our given field is nonfinite. Hence, $\mathbb{F}_{p^n}$ must have nofinite algebraic closure.
\end{proof}

\subsection{Problem 5}
\subsubsection{Question}
Exhibit an explicit isomorphism between the splitting fields of $x^3-x+1$ and $x^3-x-1$ over $\mathbb{F}_3$.
\subsubsection{Answer}
We first verify that $x+1$ is a root of $f_2(x) = x^3-x-1$ in $\mathbb{F}_3[x]/(f_2)$ since
\begin{eqnarray*}f_2(x+1) &=&(1+x)^3  -x  -2 \\
&=& x^3 +3 x^2+2 x-1\\ 
&=& x^3 +2 x-1\\ 
&=& x^3 - x-1\\ 
&=& 0
 \end{eqnarray*}
 So, the homomorphism of splitting fields $\varphi: \mathbb{F}_3[x]/(f_1) \to \mathbb{F}_3[x]/(f_2)$ defined by 
 \[\varphi: x \mapsto x+1\]
 is an isomorphism.

\subsection{Problem 7}
\subsubsection{Question}
Prove that one of $2, 3$ or $6$ is a square in $\mathbb{F}_p$ for every prime $p$. Conclude that the polynomial
\[x^6-11x^4+36x^2-36=(x^2-2)(x^2-3)(x^2-6)\]
has a root modulo $p$ for every prime $p$ but has no root in $\mathbb{Z}$.
\subsubsection{Answer}
\begin{proof}
Let $x$ be a generator of the field $\mathbb{F}_p$, and assume that neither $2$, nor $3$ are squares in $\mathbb{F}_p$. Then, since $\left<x\right> =\mathbb{F}_p$ we know that $x^l =2$ and $x^k=3$ for some $k,l \in \mathbb{Z}$. Moreover, both $k$, and $l$ must be odd, else they would be squares as $(x^{k/2})^2$ or $(x^{l/2})^2$. Hence, $x^l x^k = x^{k+l} =6$ and since $k$ and $l$ are both odd $(x^{(k+l)/2})^2=6$ and 6 is a square.\end{proof}

Since one of $2$, $3$, or $6$ is a square in $\mathbb{F}_p$ one of the corresponding polynomials $(x^2-2)$, $(x^2-3)$, or $(x^2-6)$ has a root. Hence, the product of these always has a root.

\subsection{Problem 10}
\subsubsection{Question}
Prove that $n$ divides $\varphi(p^n-1)$. [Observe that $\varphi(p^n-1)$ is the order of the group of automorphisms of a cyclic group of order $p^n-1$.]
\subsubsection{Answer}
\begin{proof}
First note that
\[\mathrm{Gal}(\mathbb{F}_{p^n}/\mathbb{F}_{p})\cong \mathbb{Z}/n \mathbb{Z} \]
which in particular implies that $|\mathrm{Gal}(\mathbb{F}_{p^n}/\mathbb{F}_{p})|  =n$. However, $\mathrm{Gal}(\mathbb{F}_{p^n}/\mathbb{F}_{p})$ is a subgroup of the group of automorphism on the additive group $\mathbb{F}_{p^n}^+$. Thus since $\varphi(p^n-1)$ is the order of the group of automorphisms of a cyclic group of order $p^n-1$ we have by Lagrange's Theorem
\[m  = \frac{ \varphi(p^n-1) }{|\mathrm{Gal}(\mathbb{F}_{p^n}/\mathbb{F}_{p})| }\]
for $m \in \mathbb{N}$ as desired.\end{proof}
\end{document}
