\documentclass[12pt]{article}
\setlength\headheight{14.5pt}
\title{Homework}
\author{Frederick Robinson\footnote{I worked with Dan Stevens on these exercises}}
\date{11 January 2010}
\usepackage{amsfonts}
\usepackage{fancyhdr}
\usepackage{amsthm}
\usepackage{setspace}
\pagestyle{fancyplain}
\doublespacing   

\begin{document}

\lhead{Frederick Robinson}
\rhead{Math 331: Algebra}

   \maketitle

\setcounter{tocdepth}{2} 

\tableofcontents

\section{Chapter 7 Section 1}
\subsection{Problem 1}

\subsubsection{Question}
Let $R$ be a ring with 1.

Show that $(-1)^2=1$ in $R$
\subsubsection{Answer}
\[1+(-1)=0\]
\[\Rightarrow (-1)(1+(-1))=(-1)0\]
\[\Rightarrow (-1)1+(-1)^2=0\]
\[\Rightarrow (-1)^2=1\]

\subsection{Problem 7}

\subsubsection{Question}
The \emph{center} of a ring $R$ is $\{z \in R\ |\ zr =rz$ for all $r\in R\}$ (i.e., is the set of all elements which commute with every element of $R$). Prove that the center of a ring is a subring that contains the identity. Prove that the center of a division ring is a field.
\subsubsection{Answer}
We will prove that the center, say $Z$ of a ring $R$ is a subring of $R$ 
\begin{proof}We need to prove  $Z$ is a subgroup of the additive group in the original ring.

Towards this end we first demonstrate that $Z$ is closed under addition. Let $x, y \in Z$. So, by the definition of $Z$ we have $x r = r x$ and $y r = r y$ for arbitrary $r \in R$. So, $x r + y r = r x + r y \Rightarrow (x+y) r= r (x+y)$.

Now we must demonstrate that the $Z$ is closed under additive inverses. Again take $x \in Z$. $r x = x r \Rightarrow -(r x) = -(x r)$ and by Proposition 1.2 (page 226) we have $-(r x) = -(x r) \Rightarrow r (-x) = (-x) r$. Thus, $Z$ is closed under additive inverses as desired. 

So we have demonstrated that $Z$ is a group under addition (as defined in the ring) if it is nonempty. 

Now we will show that multiplication as defined on $R$ is closed in $Z$. Again, we let $x, y \in Z$. Hence, $x r = r x$ and $ y r = r y$ for $r \in R$. So, $x y r = x r y = r x y  \Rightarrow (x y) r = r (x y)$ as desired.

It remains to show that $Z$ is necessarily nonempty. The additive identity of $R$ is in $Z$ as $0r = r0 = 0$ for all $r\in R$ (Proposition 1.1 page 226). If $R$ is a ring with identity then the identity is central for, $1 r = r = r 1$ for any $r \in R \Rightarrow 1\in Z$. \end{proof}

The center of a division ring is a field.
\begin{proof} By the previous proof we know that the center of a division ring $R$ is a ring. Moreover, for $R$ a division ring the center $Z$ must be a division ring since for $z\in Z, r \in R$ we have $r = z^{-1}z r = z^{-1} r z \Rightarrow r z^{-1}= z^{-1} r $.

So, since we know that $Z$ is a division ring, and every element of $Z$ commutes with every element of $R$ it must be that $\forall z,z'\in Z $ we have $z z'=z'z$. Thus, $Z$ is a commutative division ring. This is just the definition of a field though.\end{proof}

\subsection{Problem 8}
\subsubsection{Question}
Describe the center of the real Hamilton Quaternions $\mathbb{H}$. Prove that $\{a+bi\ |\ a,b\in\mathbb{R} \}$ is a subring of $\mathbb{H}$ which is a field but is not contained in the center of $\mathbb{H}$. 
\subsubsection{Answer}
First we work out the product of two elements of $\mathbb{H}$ in general.
\[(a+bi+cj+dk)(a'+b'i+c'j+d'k)\]
\[=(aa'-bb'-cc'-dd')+(ab'+ba'+cd'-dc')i+(ac'-bd'+ca'+db')j+(ad'+bc'-cb'+da')k\]

So, an element $(a+bi+cj+dk)$ is in the center of $\mathbb{H}$ if and only if we have 
\[ab'+ba'+cd'-dc'=a'b+b'a+c'd-d'c \Rightarrow cd'-dc'=c'd-d'c   \]
\[\Rightarrow  cd'=c'd   \]
and
\[ac'-bd'+ca'+db'= a'c-b'd+c'a+d'b \Rightarrow -bd'+db'= -b'd+d'b \]
\[\Rightarrow db'= d'b\]
and
\[ad'+bc'-cb'+da'=a'd+b'c-c'b+d'a \Rightarrow bc'-cb'=+b'c-c'b \]
\[\Rightarrow bc'=b'c\]
for arbitrarily chosen $a',b',c',d' \in \mathbb{R}$

An element $(a+bi+cj+dk) \in \mathbb{H}$ satisfies this if and only if it is of the form $(a+bi+cj+dk) = a$ for $a \in \mathbb{R}$. Hence, the center or $\mathbb{H}$ is just the real numbers $\mathbb{R}$.

Now we prove that $\{a+bi\ |\ a,b\in\mathbb{R} \}$ is a subring of $\mathbb{H}$ which is in particular a field
\begin{proof}The elements of $\mathbb{H}$ of the form $a+bi$ for $a,b\in \mathbb{R}$ form a group under addition since $(a+bi)+(a'+b'i)=((a+a')+(b+b')i)$ for $a,b,a',b' \in \mathbb{R}$ and $((a+a')+(b+b')i)$ is of the desired form. Moreover, given $a+bi$ we have that $a+bi+(-a-bi)=0$

Now we prove that the set of all $a+bi$ is closed under the ring product. Let $a,b,a',b' \in \mathbb{R}$. Then $(a+bi)(a'+b'i)= (aa'-bb')+(ab'+ba')i$ which is of the desired form.

So the subset in question is indeed a ring. Moreover we can show that it is a field.  It contains the identity since $1$ is of the form $a+bi$ (note that this proves also that it is nonempty). It is commutative since $(a+bi)(a'+b'i)= (aa'-bb')+(ab'+ba')i= (a'a-b'b)+(a'b+b'a)i=(a'+b'i)(a+bi)$. Lastly it is closed under multiplicative inverse since given $a+bi$ we see that $(a+bi)(\frac{a}{a^2+b^2}-\frac{b}{a^2+b^2}i)=1$ and both $\frac{a}{a^2+b^2}$ and $\frac{b}{a^2+b^2}$ are in $\mathbb{R}$.\end{proof}

$\{a+bi\ |\ a,b\in\mathbb{R} \}$ is not contained in the center of $\mathbb{H}$ since by a previous proof, the center of $\mathbb{H}$ consists exactly of those elements of form $a$ for some $a \in \mathbb{R}$


\subsection{Problem 11}

\subsubsection{Question}
Prove that if $R$ is an integral domain and $x^2=1$ for some $x\in R$ then $x=\pm 1$.
\subsubsection{Answer}
Let $R$ be an integral domain with  $x^2=1$ for some $x\in R$. We will prove that $x=\pm 1$

\begin{proof}
So, $x(x+1)=x^2+x = 1+x = x +1$. There are two cases. 

Case 1: $x+1\neq0$

In this case $x$ is the identity for $x+1$. Since identities are unique $x=1$

Case 2: $x+1=0$

So $x = -1$\end{proof}



\subsection{Problem 13}

\subsubsection{Question}
An element $x$ in $R$ is called \emph{nilpotent} if $x^m=0$ for some $m \in \mathbb{Z}^{+}$

\textbf{(a)} Show that if $n=a^k b $ for some integers $a$ and $b$ then $ab$ is a nilpotent element of $\mathbb{Z} /n\mathbb{Z}$

\textbf{(b)} If $a \in \mathbb{Z}$ is an integer, show that the element $\bar{a} \in \mathbb{Z} / n \mathbb{Z}$ is nilpotent if and only if every prime divisor of $n$ is also a divisor of $a$. In particular, determine the nilpotent elements of $\mathbb{Z} / 72\mathbb{Z}$ explicitly.

\textbf{(c)} Let $R$ be the ring of functions from a nonempty set $X$ to a field $F$. Prove that $R$ contains no nonzero nilpotent elements.

\subsubsection{Answer}

\textbf{(a)} Let $n=a^k b $ for some integers $a$ and $b$. We will show that  $ab$ is a nilpotent element of $\mathbb{Z} /n\mathbb{Z}$

\begin{proof} $(ab)^k = a^k b^k = (a^k b) b^{k-1} = n b^{k-1} \equiv 0 (mod\ n) $\end{proof}

\textbf{(b)}  Let $a \in \mathbb{Z}$. We will show that the element $\bar{a} \in \mathbb{Z} / n \mathbb{Z}$ is nilpotent if and only if every prime divisor of $n$ is also a divisor of $a$.


\begin{proof}
We begin by showing $(\Rightarrow)$ that $\bar{a} \in \mathbb{Z} / n \mathbb{Z}$ is nilpotent if every prime divisor of $n$ is also a divisor of $a$.

If we assume every prime divisor of $n$ is also a divisor of $a$ we have $n=p_1^{\alpha_1}\cdot p_2^{\alpha_2}\cdot \dots \cdot p_m^{\alpha_m}$ and $a=p_1^{\beta_1}\cdot p_2^{\beta_2}\cdot \dots \cdot p_m^{\beta_m}\cdot (l)$ for $\alpha_1, \dots, \alpha_m, \beta_1, \dots \beta_m, l \in \mathbb{N} \backslash 0$ and $p_1, \dots, p_m$ distinct primes.

Let $A = max\{\alpha_1, \dots , \alpha_m\}$. We have then that $a^A= (p_1^{\beta_1}\cdot p_2^{\beta_2}\cdot \dots \cdot p_m^{\beta_m}\cdot (l))^A= p_1^{A \beta_1}\cdot p_2^{A \beta_2}\cdot \dots \cdot p_m^{A \beta_m}\cdot (l^A)$ and since each $\beta_i$ is at least $1$ we can express this as $a^A =p_1^{\alpha_1}\cdot p_2^{\alpha_2}\cdot \dots \cdot p_m^{\alpha_m} \cdot p_1^{\gamma_1}\cdot p_2^{\gamma_2}\cdot \dots \cdot p_m^{\gamma_m}\cdot (l^A) =  n \cdot p_1^{\gamma_1}\cdot p_2^{\gamma_2}\cdot \dots \cdot p_m^{\gamma_m}\cdot (l^A) \equiv (0$ mod $n)$. Thus, $a$ is nilpotent.

Now we show $(\Leftarrow)$ that if $\bar{a} \in \mathbb{Z} / n \mathbb{Z}$ is nilpotent then every prime divisor of $n$ is also a divisor of $a$. 

Assume that $\bar{a} \in \mathbb{Z} / n \mathbb{Z}$ is nilpotent. Then, $\exists A$ such that $a^A = n l $ for some $l \in \mathbb(Z)^{+}$. However this is only possible if every prime divisor of $n$ is also a divisor of $a$. For, assume not, then there is some divisor $m$ of $n$ that is not a divisor of $a$. But, this is a contradiction as $a^A$ is divisible only by those primes which divide $a$. 

So we have shown  that the element $\bar{a} \in \mathbb{Z} / n \mathbb{Z}$ is nilpotent if and only if every prime divisor of $n$ is also a divisor of $a$ as desired. \end{proof}

The nilpotent elements of $\mathbb{Z} / 72\mathbb{Z}$  are precisely those whose representatives have the same prime divisors as $72$ by the preceding proof. 

Since $72=2^3 \cdot 3^2$ the representatives in $[0,72]$ are just $6, 12, 18, 24, 30, 36, 42,$ $48, 54, 60, 66, 72$, and so the nilpotent elements are $\bar{6}, \bar{12}, \bar{18}, \bar{24}, \bar{30}, \bar{36}, \bar{42}, \bar{48},$ $\bar{54}, \bar{60}, \bar{66}, \bar{72}$

\textbf{(c)} Let $R$ be the ring of functions from a nonempty set $X$ to a field $F$. We shall prove that $R$ contains no nonzero nilpotent elements.

\begin{proof} Suppose towards a contradiction that $R$ contains a nonzero nilpotent element, say $f: X \to F$. So $f\neq0$ and $f^k=0$ for some $k\in \mathbb{N}$. However, this implies that for each $x \in X$ we have $f(x)^k=0$, as the additive identity in $R$ is that function which takes each $x\in X$ to $0$. Since $f$ is nonzero we can choose this $f(x)$ to be nonzero.

So, $f(x) \in F$ is nonzero and nilpotent. This is however a contradiction, for it implies that $f(x)^{k-1}\cdot f(x)= 0 \Rightarrow f(x)$ is a zero divisor. But, fields have no zero divisors.
\end{proof}

\subsection{Problem 14}

\subsubsection{Question}
Let $x$ be a nilpotent element of the commutative ring $R$ (cf. the preceding exercise).

\textbf{(a)} Prove that $x$ is either zero or a zero divisor.

\textbf{(b)} Prove that $rx$ is nilpotent for all $r \in R$. 

\textbf{(c)} Prove that $1+x$ is a unit in $R$.

\textbf{(d)} Deduce that the sum of a nilpotent element and a unit is a unit.


\subsubsection{Answer}

\textbf{(a)} We shall show that $x$ is either zero or a zero divisor.
\begin{proof}
Since $x$ is nilpotent, $\exists k\in \mathbb{N}$ such that $x^k=0$. There are two cases.

Case 1: $k=1$

$x^1=0 \Rightarrow x=0$

Case 2: $k \neq 1$

$x\cdot x^{k-1}=0$. This is just the definition for $x$ being a zero divisor though as $x^{k-1}\neq 0$ for $k$ minimal.\end{proof}


\textbf{(b)} We prove that $rx$ is nilpotent for all $r \in R$. 
\begin{proof}
Since $x$ is nilpotent there is some $k \in \mathbb{N}$ such that $x^k=0$. Moreover, since $R$ is commutative we have that $(rx)^k=r^k x^k= r^k \cdot 0 = 0$. Hence, $rk$ is nilpotent.
\end{proof}

\textbf{(c)} Now we show that $1+x$ is a unit in $R$.

\begin{proof}
\[(1+x)(1-x+x^2-x^3+\dots\pm x^{k-1})=1-x+x^2-x^3+\dots\pm x^{k-1}+x-x^2+x^3-x^4+\dots\pm x^{k}\]
\[=1+x^k\]
\[=1\]
So, $1+x$ is a unit in $R$ is desired.\end{proof}

\textbf{(d)} Deduce that the sum of a nilpotent element and a unit is a unit.

\begin{proof}
Let $x, a \in R$ with $x$ nilpotent and $a$ a unit. Then we have 
\begin{eqnarray*}
a^{-1}(x+a)&=&a^{-1}x+a^{-1}a\\
&=&a^{-1}x +1
\end{eqnarray*}
Which is a unit as, we know by part b that $rx$ is a nilponent element, and adding the identity to a nilponent element yields a unit by c. Moreover the fact that multiplying $(x+a)$ by $a^{-1}$ yileds a unit proves that $(x+a)$ is a unit because 
\[\left(a^{-1}(x+a)\right)^{-1}a^{-1}(x+a)=1\Rightarrow \left(\left(a^{-1}(x+a)\right)^{-1}a^{-1}\right)(x+a)=1\]
\end{proof}


\subsection{Problem 17}

\subsubsection{Question}
Let $R$ and $S$ be rings. Prove that the direct product $R \times S$ is a ring under componentwise addition and multiplication. Prove that $R \times S$ is commutative if and only if both $R$ and $S$ are commutative. Prove that $R \times S$ has an identity if and only if both $R$ and $S$ have identities. 

\subsubsection{Answer}
To prove that $R \times S$ is a ring under componentwise addition and multiplication we must prove first that it is a nonempty subgroup under addition, then that it is closed under multiplication.

Let $(r,s)$ and $(r',s')$ be elements of $R \times S$
\begin{proof}
\[(r,s)+(r',s')=(r+r',s+s')\]
Thus, the group is closed under addition by closure of addition on $R, S$. Moreover we have closure under additive inverses as $(r,s)+(-r,-s)=(0,0)$ for any $r,s$. Lastly we know that the group is nonempty for it contains at least the element $(0,0)$ as each of the groups $R, S$ have additive identities.

The group is closed under multiplication since $R,S$ are closed under multiplication that is $(r,s)(r',s')=(rr',ss')$ and $(rr',ss')\in R\times S$ by closure of $R,S$ under multiplication.\end{proof}

$R \times S$ is commutative if and only if $R,S$ are both commutative
\begin{proof}
Let $r,r' \in R$ and $s,s' \in S$.

First we prove ($\Rightarrow$) that $R \times S$ is commutative if $R$ and $S$ are both commutative. 

\[(r,s)(r',s')=(rr',ss')=(r'r,s's)=(r',s')(r,s)\]

Now we prove ($\Leftarrow$) that if $R \times S$ is commutative then both $R$ and $S$ are commutative. For if $R \times S$ commute then in particular the elements of form $(\bar{r},s)$ and $(r,\bar{s})$ (for $\bar{r}$ and $\bar{s}$ fixed elements of $R$ and $S$ respectively) commute with every other element. This implies that $S$ and $R$ respectively are commutative as $(r,\bar{s})(r',\bar{s})=(r',\bar{s})(r,\bar{s}) \Rightarrow rr'=r'r$ and $(\bar{r},s)(\bar{r},s')=(\bar{r},s')(\bar{r},s) \Rightarrow ss'=s's$

Thus we have established that $R \times S$ is commutative if and only if $R$ and $S$ are both commutative as desired.\end{proof}

We next prove that $R \times S$ has identity if and only if both $R$ and $S$ have identity.

\begin{proof}
We first prove ($\Rightarrow$) that $R \times S$ has identity if $R$ and $S$ both have identity.

$(1_R,1_S)$ is the identity of $R \times S$ as for arbitrary $r \in R, s \in S$ we have $(1,1)(r,s)=(r,s)=(r,s)(1,1)$

Now we prove that if $(\hat{r},\hat{s})$ is the identity in $R \times S$ then $\hat{r}$ and $\hat{s}$ are the identities in $R$ and $S$ respectively

Given arbitrary $r \in R, s \in S$ we have $(\hat{r},\hat{s})(r,s)=(\hat{r}r,\hat{s}s)=(r,s)=(r\hat{r},s\hat{s})=(r,s)(\hat{r},\hat{s})$ so it follows that $\hat{r}$ and $\hat{s}$ are identities in $R$ and $S$ respectively as claimed.\end{proof}

\subsection{Problem 18}

\subsubsection{Question}
Prove that $\{ (r,r) \ |\ r\in R\}$ is a subring of $R \times R$.
\subsubsection{Answer}

 $\{ (r,r) \ |\ r\in R\}$ is a subring of $R \times R$.
 \begin{proof}
 This set forms an additive group since given $r,r' \in R$ we have $(r,r)+(r',r')=(r+r',r+r')$ and $r+r' \in R$ by closure of $R$ under addition. Moreover $R \times R$ is closed under additive inverses since $(r,r)+(-r,-r)=(0,0)$ and $-r$ is in $R$ for each $r$ by closure of $R$ under additive inverses. Also we know that $R \times R$ is nonempty since $R$ contains $0$ by group properties, implying that $(0,0)\in R \times R$ 
 
 It remains to confirm that $R \times R$ is closed under the multiplication operation. So, similar to the above we observe that this follows from the same property of $R$. That is, $(r,r)(r',r')=(rr',rr')$ and $rr'$ is in $R$ by closure of $R$ under multiplication. 
 
 Finally we note that throughout the above each sum, inverse, product is of the desired form: $(r,r)$ for $r \in R$. Hence, we have shown that $(r,r)$ is a subring of $R \times R$ as desired.\end{proof}

\end{document}
