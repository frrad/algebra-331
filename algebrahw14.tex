\documentclass[10pt]{article}
\setlength\headheight{14.5pt}
\title{Homework}
\author{Frederick Robinson}
\date{14 April 2010}
\usepackage{amsfonts}
\usepackage{fancyhdr}
\usepackage{amsthm}
\usepackage{setspace}
%\doublespacing
\pagestyle{fancyplain}

\begin{document}

\lhead{Frederick Robinson}
\rhead{Math 331: Algebra}

   \maketitle

\setcounter{tocdepth}{2} 

\tableofcontents

\section{Chapter 13 Section 4}
\subsection{Problem 1}
\subsubsection{Question}
Determine the splitting field and its degree over $\mathbb{Q}$ for $x^4-2$.
\subsubsection{Answer}
Note that
\[f(x)=x^4-2=(x-\sqrt[4]2)(x+\sqrt[4]2)(x+i\sqrt[4]{2})(x-i\sqrt[4]{2})\]
So, the splitting field is just $\mathbb{Q}(\sqrt[4]2,i)$. This is a field extension of degree 8 over $\mathbb{Q}$ since the degree of $\mathbb{Q}(\sqrt[4]2)$ is 4 (minimal polynomial $f$), the degree of $\mathbb{Q}(i)$ is 2 and these extensions have nothing in common (in particular $\mathbb{Q}(\sqrt[4]2) - \mathbb{Q} \subset \mathbb{R}$ and $\mathbb{R} \cap (\mathbb{Q}(i)- \mathbb{Q}) = \emptyset$).

\subsection{Problem 2}
\subsubsection{Question}
Determine the splitting field and its degree over $\mathbb{Q}$ for $x^4+2$.
\subsubsection{Answer}
Again we note that
\[f(x)=x^4+2=(x+\sqrt[4]{-2})(x+i \sqrt[4]{-2})(x-i \sqrt[4]{-2})(x-\sqrt[4]{-2})\]
The splitting field is then $\mathbb{Q}(\sqrt[4]{-2},i)$. The degree of this extension is 8 since $\mathbb{Q}(\sqrt[4]{-2})$ is a degree 4 extension (minimal polynomial $f$) and $\mathbb{Q}(i)$ is a degree 2 extension over $\mathbb{Q}(\sqrt[4]2)$.
\subsection{Problem 3}
\subsubsection{Question}
Determine the splitting field and its degree over $\mathbb{Q}$ for $x^4+x^2+1$.
\subsubsection{Answer}
Again we note that
\[f(x)=x^4+x^2+1=(x+(-1)^{1/3})(x-(-1)^{1/3})(x+(-1)^{2/3})(x-(-1)^{2/3})\]
The splitting field is then $\mathbb{Q}((-1)^{1/3})$ which has degree 3 over $\mathbb{Q}$ (minimal polynomial $g(x)=x^3+1$).

\subsection{Problem 5}
\subsubsection{Question}
Let $K$ be a finite extension of $F$. Prove that $K$ is a splitting field over $F$ if and only if every irreducible polynomial in $F[x]$ that has a root in $K$ splits completely in $K[x]$. [Use Theorems 8 and 27.]
\subsubsection{Answer}
\begin{proof}

Say $K$ is a splitting field for some polynomial $f \in F[x]$ over $f$ and $p(x) \in F[x]$ is an irreducible polynomial with roots in $K$ say $\alpha$. Let $D$ be the splitting field of $p(x)$ over $K$ and $\beta$ be any root. $F(\alpha)\cong F[x]/(p(x))\cong F(\beta)$ so by Theorem 27 this extends to an isomorphism of splitting fields $\sigma$. Since $K$ is the splitting field of $f$ we have $\sigma(K)=K$. In particular $\sigma(\alpha)=\beta=K$, and $p(x)$ splits over $K$.

Conversely if every irreducible polynomial in $F[x]$ with a root in $K$ splits over $F$ there exist a set of $\alpha_1, \alpha_2, \dots, \alpha_n \in K$ which generate $K$ over $F$. If we denote the minimal polynomial for each $\alpha_i$ by $p_i$ then $K$ is the splitting field of $p_1p_2 \dots p_n$ over $F$.

\end{proof}

\section{Chapter 13 Section 5}
\subsection{Problem 2}
\subsubsection{Question}
Find all irreducible polynomials of degrees $1,2$ and $4$ over $\mathbb{F}_2$ and prove that their product is $x^{16}-x$
\subsubsection{Answer}
An exhaustive search reveals that the following are all irreducible polynomials of degree $1,2,4$ over $\mathbb{F}_2$
\[x,\  x+1,\  x^2 + x+ 1,\  x^4+x+1,\  x^4+x^3+1,\  x^4+x^3+x^2+x+1  \]
Moreover we compute the product of all the above to yield
\begin{eqnarray*}x+4 x^2+8 x^3+12 x^4&+&18 x^5+26 x^6+32 x^7+34 x^8+34 x^9\\&+&32 x^{10}+26 x^{11}+18 x^{12}+12 x^{13}+8 x^{14}+4 x^{15}+x^{16}\end{eqnarray*}
which, reducing coefficients mod 2 is just
\[x+x^{16}\]
as desired.
\subsection{Problem 3}
\subsubsection{Question}
Prove that $d$ divides $n$ if and only if $x^d-1$ divides $x^n-1$. [Note that if $n=q d +r$ then $x^n-1=(x^{qd+r}-x^r)+(x^r-1)$.]
\subsubsection{Answer}
\begin{proof}
$x^d-1$ divides $x^n-1$ if and only if every root of $x^d-1$ is also a root of $x^n-1$. In particular then $x^d-1$ divides $x^n-1$ if and only if $x^n=1$ for every $x$ such that $x^d=1$. Writing $n=qd+r$ we see that $x^n=x^{qd+r}={(x^d)}^q x^r$, so $x^d-1$ divides $x^n-1$ if and only if $r=0$: that is, if and only if $d$ divides $n$.
\end{proof}

\subsection{Problem 4}
\subsubsection{Question}
Let $a>1$ be an integer. Prove for any positive integers $n$, $d$ that $d$ divides $n$ if and only if $a^d-1$ divides $a^n-1$ (cf. previous exercise). Conclude in particular that $\mathbb{F}_{p^d} \subseteq \mathbb{F}_{p^n}$ if and only if $d$ divides $n$.
\subsubsection{Answer}
\begin{proof}
By the previous if $d$ divides $n$ if and only if $a^d-1$ divides $a^n-1$. .

So, since $\mathbb{F}_{p^d} \subseteq \mathbb{F}_{p^n}$ if and only if $a^d-1$ divides $a^n-1$, $\mathbb{F}_{p^d} \subseteq \mathbb{F}_{p^n}$ if and only if $d$ divides $n$.
\end{proof}


\end{document}
